\documentclass[]{article}
\usepackage{etex}
\usepackage[margin = 1.5in]{geometry}
\setlength{\parindent}{0in}
\usepackage{amsmath}
\usepackage{amsfonts}
\usepackage{amssymb}
\usepackage{amsthm}
\usepackage{listings}
\usepackage{color}
\usepackage{mathtools}
\usepackage{multicol}
\usepackage[lined]{algorithm2e}
\usepackage{float}
\usepackage[T1]{fontenc}
\usepackage{ae,aecompl}
\usepackage[pdftex,
pdfauthor={Michael Noukhovitch},
pdftitle={},
pdfsubject={Lecture notes from },
pdfproducer={LaTeX},
pdfcreator={pdflatex}]{hyperref}

\usepackage{cleveref}
\usepackage{enumitem}

\definecolor{dkgreen}{rgb}{0,0.6,0}
\definecolor{gray}{rgb}{0.5,0.5,0.5}
\definecolor{mauve}{rgb}{0.58,0,0.82}
\DeclareMathOperator*{\argmax}{arg\,max}
\DeclareMathOperator*{\argmin}{arg\,min}

\lstset{
    language=C,
    aboveskip=3mm,
    belowskip=3mm,
    showstringspaces=false,
    columns=flexible,
    basicstyle={\small\ttfamily},
    numbers=none,
    numberstyle=\tiny\color{gray},
    keywordstyle=\color{blue},
    commentstyle=\color{dkgreen},
    stringstyle=\color{mauve},
    breaklines=true,
    breakatwhitespace=true,
    tabsize=4
}

\theoremstyle{definition}
\newtheorem*{defn}{Definition}
\newtheorem{ex}{Example}[section]
\newtheorem*{theorem}{Theorem}

\setlength{\marginparwidth}{1.5in}
\setlength{\algomargin}{0.75em}

\DeclarePairedDelimiter{\set}{\lbrace}{\rbrace}

\definecolor{darkish-blue}{RGB}{25,103,185}

\usepackage{hyperref}
\hypersetup{
    colorlinks,
    citecolor=darkish-blue,
    filecolor=darkish-blue,
    linkcolor=darkish-blue,
    urlcolor=darkish-blue
}
\newcommand{\lecture}[1]{\marginpar{{\footnotesize $\leftarrow$ \underline{#1}}}}

\makeatletter
\def\blfootnote{\gdef\@thefnmark{}\@footnotetext}
\makeatother

\begin{document}
\let\ref\Cref

\title{\bf{Natural Language Processing}}
\date{Fall 2020, McGill\\ \center Notes written from Jackie Cheung's lectures}
\author{Michael Noukhovitch}

\maketitle
\newpage

\tableofcontents

\newpage

\section{Introduction}%
\label{sec:introduction}

\subsection{Overview}%
\label{sub:overview}

\textbf{language} is a form of communication
\begin{itemize}
    \item \textit{arbitrary} pairing between form and meaning
    \item very expressive and productive
    \item nearly universal
    \item uniquely human*
\end{itemize}

\textbf{computational linguistics} modelling natural language with computational models
\begin{itemize}
    \item acoustic signals
    \item NL understanding (comprehension)
    \item NL generation (production)
\end{itemize}

goals of the field
\begin{itemize}
    \item practical technologies (NLP)
    \item understanding how language works (CL)
\end{itemize}

models and techniques
\begin{itemize}
    \item gathering data
    \item evaluation
    \item statistical methods (ML)
    \item rule-based systems
\end{itemize}

some example problems
\begin{itemize}
    \item is language an instinct? (Chomsky)
    \item language processing to understand meaning of sentence
    \item can we learn mathematical properties of language
\end{itemize}

types of language
\begin{itemize}
    \item \textbf{text} an idealization of spoken language
        \begin{itemize}
            \item luckily English is similar between writing and speaking, and there is lots of data on it
            \item older work used ``clean'' language but recent work ventures into messy data (e.g. Twitter)
        \end{itemize}
    \item \textbf{speech} is much messier
        \begin{itemize}
            \item automatic speech recognition (ASR)
            \item text-to-speech generation (TTS)
        \end{itemize}
\end{itemize}

\subsection{Domains of Language}%
\label{sub:domains_of_language}

\textbf{phonetics} study of speech sounds
\begin{itemize}
    \item articulation, transmission
    \item how each sound is made in the mouth
\end{itemize}
\textbf{phonology} rules that govern sound patterns
\begin{itemize}
    \item how the sounds are organized
    \item ``p'' in peach and speach are the same phoneme but phonetically distinct (aspiration)
\end{itemize}
\textbf{morphology} word formation and meaning
\begin{itemize}
    \item anti-dis-establish-ment-arian-ism
\end{itemize}
\textbf{syntax} structure of language
\begin{itemize}
    \item ``I a woman saw park in the'' is \textbf{ungrammatical}
    \item \textbf{ambiguity} different possible meaning for the same phrase
\end{itemize}
\textbf{semantics} meaning of language
\begin{itemize}
    \item ``Ross wants to marry \textbf{a} Swedish woman''
\end{itemize}
\textbf{pragmatics} meaning of language in context
\begin{itemize}
    \item different from literal meaning
    \item \textbf{deixis} interpretation that relies on extra-linguistic context
    \item ``dessert would be delicious''
\end{itemize}
\textbf{discourse} structure of larger spans of language
\begin{itemize}
    \item do large spans of text form a coherent story
\end{itemize}

\subsection{Technology}%
\label{sub:technology}

combination of hand-crafted knowledge and ML on data
\begin{itemize}
    \item rule-based systems
    \item machine learning
    \item knowledge representation
\end{itemize}

\section{Text Classification}%
\label{sec:text_classification}

\subsection{Basics}%
\label{sub:basics}

\textbf{text classification} assign a label or category to a piece of text
\begin{itemize}
    \item sentiment analysis
    \item spam detection
    \item language identification
    \item authorship attribution
\end{itemize}

\textbf{supervised} output data is labelled
\begin{itemize}
    \item learn a function, minimize $\theta$ with loss on data
    \item e.g. spam classificaiton, predict POS
    \item \textbf{regression} $y$ is continuous
    \item \textbf{classification} $y$ is discrete
\end{itemize}
\textbf{unsupervised} output data is unlabelled
\begin{itemize}
    \item learn a density
    \item e.g. grammar induction, word-relatedness (word2vec)
\end{itemize}

\subsection{Building a text classifier}%
\label{ssub:building_a_text_classifier}

\begin{itemize}
    \item define problem, collect data
    \item extract feats
    \item train a classifier on train data
    \item apply classifier to test data
\end{itemize}

problem definition
\begin{itemize}
    \item problem
    \item input
    \item output categories
    \item how to annotate
\end{itemize}

\subsection{Feature Extraction}%
\label{ssub:feature_extraction}


\textbf{feature extraction} get ``important'' properties of documents
\begin{itemize}
    \item convert text into numerical format
    \item e.g. word counts as features \textit{unigram counts}
\end{itemize}

\textbf{lemma} remove affixes get dictionary word ``flies $\to$ fly''
\textbf{stemming} remove affix get stem ``airliner $\to$ airlin''
\begin{itemize}
    \item rule-based e.g. (Porter, 1980) ``ies $\to$ i''
\end{itemize}
\textbf{n-grams} sequences of adjacent words
\begin{itemize}
    \item presence or absence
    \item counts
    \item proportion of total document
    \item scaled version (tf-idf)
\end{itemize}
\textbf{POS tags} crudely capture syntactic pattern (PTB dataset)
\textbf{stop-word removal} remove common uninformative words

sentence representation
\begin{itemize}
    \item vector addition $v_{hello} + v_{world}$
    \item vector multiplication $v_{hello} \cdot v_{world}$
    \item \textbf{max pooling} choose the max of vectors
\end{itemize}


\subsection{Models}%
\label{sub:models}

\textbf{training} select parameters $\theta^*$ according to some objective

types of models
\begin{itemize}
    \item \textbf{generative} models joint distribution $P(x, y)$
        \begin{itemize}
            \item less flexible features as they need to be consistent with each other
        \end{itemize}
    \item \textbf{discriminative} models conditional $P(y|x)$
        \begin{itemize}
            \item can be more flexible in terms of features
        \end{itemize}
\end{itemize}

\subsubsection{Naive Bayes}%
\label{ssub:naive_bayes}

\textbf{Naive Bayes} probabilistic classifier that uses Bayes' Rule $P(y|x) = \frac{P(y)P(x|y)}{P(x)}$
\begin{itemize}
    \item generative
    \item assumes data $x$ is generated independently conditioned on class $P(x_i | y)$
    \item graphical assumption $P(x,y) = P(y) \prod_i P(x_i|y)$
\end{itemize}

In NLP, we can assume NB over a \textit{categorical} distribution and train
\begin{itemize}
    \item loss $L = \prod_{(x,y) \in D} P(y) \prod_i P(x_i|y)$
    \item learn $P(Y=y)$ proportion of samples with class $y$
    \item learn $P(X_i = x| Y= y)$ proportion of samples with feature $x$ given class $y$
\end{itemize}

Inference time we want $P(y|x)$
\begin{align}
    P(y|x) &= P(x,y) | P(x) \\
           &= P(y) \prod_i P(x_i|y) / P(x)
\end{align}
where $P(x)$ is the marginalized over all classes

\subsubsection{Logistic Regression}%
\label{ssub:logistic_regression}

\textbf{logistic regression} linear regression with a logit activation
\begin{itemize}
    \item $P(y|x) = \frac{1}{Z} \exp (\sum_i a_i x_i)$
    \item squash output between $(0,1)$
\end{itemize}

train log-likelihood with \textit{gradient descent}

\begin{align}
    \log L(\theta) &= \prod_{(x,y) \in D} \log P(y|x;\theta) \\
                   &=  \prod_{(x,y) \in D} (\sum_i a_i x_i - \log Z)
\end{align}

\subsubsection{Support Vector Machines}%
\label{ssub:support_vector_machines}

SVM learns linear decision to maximize margin to nearest sample in each of two classes
\begin{itemize}
    \item can be non-linear using \textit{kernels}
\end{itemize}

\subsubsection{Neural Network}%
\label{ssub:neural_network}

\textbf{Perceptron} logistic regression with Perceptron learning rule
$f(x) = \begin{cases} 1 & \text{if } wx+b > 0 \\ 0 & \text{else} \end{cases}$

\textbf{Stacked Perceptron} stacks perceptron neurons

\textbf{Artificial Neural Network} stacked neurons with non-linear activation functions
\begin{itemize}
    \item can learn complex functions
    \item need lots of data and computational power
    \item given enough neurons can compute any function
    \item highly flexible, generic architecture
    \item \textbf{multi-task learning} train model to solve multiple tasks simultaneously
    \item \textbf{transfer learning} use a network on a task different from its training
\end{itemize}

\textbf{Feed-forward} network
\begin{align}
    h_1 &=  g_1(W_1 x + b_1) \\
    h_2 &=  g_2(W_2 h_1 + b_2) \\
    y &= W_2 h_2
\end{align}

\textbf{Activation function} on the output of a neuron
\begin{itemize}
    \item sigmoid
    \item tanh
    \item ReLU
    \item softmax $\frac{exp(x_i)}{\sum_j \exp(x_j)}$
\end{itemize}

Training neural network done with \textbf{gradient descent} on loss function
\begin{itemize}
    \item \textbf{backpropogation} (Rumelhart et al, 1986) uses chain rule to pass gradient
    \item \textbf{SGD} apply gradient iteratively over mini-batches of data
\end{itemize}


\subsection{Model Selection}%
\label{sub:model_selection}

How to choose preprocessing, model, etc.. evaluate on unseen data!

Data split
\begin{itemize}
    \item \textbf{training} learning the model, 60-90\%
    \item \textbf{dev/validation} evaluating while learning the model
    \item \textbf{testing} evaluate once at the end to see how well you do
\end{itemize}

\textbf{k-fold cross-validation} split training data into $k$ folds, train on $k-1$ fold and test on the last

Evaluation measures
\begin{itemize}
    \item \textbf{accuracy} correct / total samples
    \item \textbf{precision} true positives / all predicted positives
    \item \textbf{recall} true positives / true number of positives
    \item \textbf{F1} $\frac{2 * P * R}{P+R}$, can average with P, R, F1
\end{itemize}

You can average evaluation method across classes
\begin{itemize}
    \item \textbf{macro-average} takes the average after computing P, R, F1 for each class
        \begin{itemize}
            \item $\frac{P_{spam} + P_{not-spam}}{2}$
            \item weighs each \textit{class} equally
        \end{itemize}
    \item \textbf{micro-average} sum of counts first then compute P, R, F
        \begin{itemize}
            \item $P_{micro} = \frac{TP_{spam} + TP_{not-spam}}{AP_{spam} + AP_{not-spam}}$
            \item weighs each \textit{sample} equally
        \end{itemize}
\end{itemize}

key issues
\begin{itemize}
    \item which eval measure to use
    \item stastical significance of test
    \item do these tests matter?
\end{itemize}

\section{Language Modelling}%
\label{sec:language_modelling}

\subsection{Words}%
\label{sub:words}

\textbf{word} smallest unit that can appear in isolation
\begin{itemize}
    \item philosophically not so clear e.g. ``peanut butter'' vs ``football''
    \item practically: \textit{spaces delimit words}
\end{itemize}

word features
\begin{itemize}
    \item \textbf{type} identity of a word (count each word once)
    \item \textbf{token} instance of a word (count number of occurences)
\end{itemize}

counting words
\begin{itemize}
    \item \textbf{term frequency} $TF(w,S) = $number of $w$ in corpus $S$
    \item \textbf{relative frequency} $RF(w,S) = \frac{TF(w,S)}{|S|}$
\end{itemize}

\textbf{corpus} of text to use, e.g.
\begin{itemize}
    \item Brown
    \item British National Corpus
    \item WSJC
\end{itemize}

\textbf{Zipf's law} on the ``long tail'' $f \propto \frac{1}{r}$
\begin{itemize}
    \item frequency of  word is inversely proportional to its rank (by frequency in the corpus)
    \item 100th most common word appears 100x less than most common word
    \item \textbf{Zipf-Mandelbrot} $f = \frac{P}{(r + \rho)^B}$
    \item proportion differs by language
\end{itemize}


\subsection{N-gram Language Models}%
\label{sub:ngram_language_models}

\textbf{language modelling} predicting the next word in a context $P(W = w| C)$
\begin{itemize}
    \item can break down with chain rule $P(w_1,2) = P(w_2|w_1)P(w_1)$
    \item should assign higher probability to grammatical sentence
    \item shouldn't be used to predict grammaticality
    \item captures linguistic knowledge and maybe facts about world
\end{itemize}

\textbf{n-grams} feature using $N-1$ previous words as context using MLE
\begin{itemize}
    \item \textbf{unigram} 1 word, $P(cats) = \frac{\text{count(cats)}}{\text{count(all words}}$
    \item \textbf{bigram} 2 word $P(cats | the) = \frac{\text(counts(the cats)}{\text{count(the)}}$
    \item \textbf{trigram} $N=3$ word context
\end{itemize}

evaluation measure
\begin{itemize}
    \item \textbf{likelihood} of generating the test corpus
    \item \textbf{cross-entropy} $H(p, q) = - \sum_i p_i \log_2 q_i$ with model distribtuion $q$
        \begin{itemize}
            \item information $I(x) = \log_2 \frac{1}{P(x)}$
            \item entropy $H(p) = -\sum_i p_i \log_2 p_i$
            \item approx cross entropy with $H(p,q) = -\frac{1}{N} \log_2 q(w_1,\ldots, w_N)$
        \end{itemize}
    \item \textbf{perplexity} $2^{H(p,q)}$ to make differences larger
\end{itemize}


\subsubsection{Learning}%
\label{ssub:learning}


\textbf{maximum likelihood estimation} choose parameters $\theta$ to maximize likelihood training corpus $X$
\begin{itemize}
    \item i.i.d assumption of words in corpus $P(C;\theta) = \prod_n P(x_n;\theta)$
    \item for a Bernoulli $P(C;\theta) = \theta^{N_1}(1-\theta)^{N_0} \ldots \theta_{MLE} = \frac{N_1}{N_0 + N_1}$
\end{itemize}

two issues with learning
\begin{itemize}
    \item \textbf{overfitting} with high train accuracy, low test data accuracy
    \item \textbf{underfitting} low train accuracy
\end{itemize}

model complexity tradeoff
\begin{itemize}
    \item highly expressive models overfit easier
    \item choose a model class and regularization/smoothing technique
\end{itemize}


\subsubsection{Smoothing}%
\label{ssub:smoothing}

out-of-vocabulary (OOV) at test time is a real problem

simple solution
\begin{itemize}
    \item replace all words less than freq threshold with \texttt{<UNK>}
    \item treat \texttt{<UNK>} as a vocabulary item
\end{itemize}

\textbf{smoothing} probability distribution, move mass to unseen cases
\begin{itemize}
    \item don't trust frequency counts entirely
    \item no longer doing MLE but can use $\theta_{MLE}$ prior
    \item $\theta_{smooth} = \max_\theta P(X;\theta)P(\theta) $
\end{itemize}

\textbf{add-$\delta$ smoothing} add frequency to each word ( \textit{pseudocount}(
\begin{itemize}
    \item for unigram $P(w) = \frac{count(w) + \delta}{N+\delta *|Lexicon|}$
    \item with $\delta=1$ called \textbf{laplace discounting}
\end{itemize}

\textbf{interpolation} lower $N$ in $N$-gram to alleviate data sparsity
\begin{itemize}
    \item combine $\hat P = \lambda_1 P_{unigram}^{MLE} + \lambda_2 P_{bigram}^{MLE} + \lambda_3 P_{trigram}^{MLE}$
\end{itemize}

\textbf{Good-Turing Smoothing} assume probability based on zipf's law, assuming all UNK appear once and readjust others accordingly
\begin{itemize}
    \item build a histogram of event frequency $f_c$ (e.g. 1000 words appear once $f_1 = 1000$, 600 words appear twice $f_2 = 600$)
    \item total number of observed event-tokens $N = \sum_i f_i * i$
    \item $w_c$ is event that occurs $c$ times
    \item choose probability for all unseen $P(UNK) = \frac{f_1}{N}$
    \item readjust other mass $c^* = \frac{(c+1)f_{c+1}}{f_c},P(w_c) = \frac{c^*}{N}$
\end{itemize}

\textbf{GT-refinement} estimate linear regression $f_c^{LR} = a \log c + b$ for higher values of $c$
\begin{itemize}
    \item Good-Turing fails for higher $c$ where $f_{c+1} = 0$
    \item use linear regression estimate for $c >$ threshold
    \item use regular $f_c$ for $c < $ threshold
\end{itemize}

\subsection{Hidden Markov Models}%
\label{ssec:hidden_markov_models}

\subsubsection{POS Tagging}%
\label{ssub:pos_tagging}

\textbf{part of speech} syntactic category that tells you grammatical properties of a word
\begin{itemize}
    \item nouns
    \item verbs
    \item adjectives
    \item prepositions
    \item adverbs
    \item determiners ``the, a, an''
\end{itemize}

other POS
\begin{itemize}
    \item modals and auxiliary verbs ``*did* you see him?''
    \item conjunctions ``and, or, but, yet''
    \item particles ``look *up* and *down*''
\end{itemize}

classes of POS
\begin{itemize}
    \item open class: where new words (\textbf{neologism}) can be added to the language
        \begin{itemize}
            \item e.g. nouns like "Kleenex", adjectives like "sick"
            \item tend to be content words
        \end{itemize}
    \item closed class: new words \textit{tend} not to be added
        \begin{itemize}
            \item function words that convey grammatical info
        \end{itemize}
\end{itemize}

corpora
\begin{itemize}
    \item Penn Treebank (45 tags)
    \item Brown corpus (87 tags)
\end{itemize}

language differences
- japanese doesn't differ between nouns and pronouns but verbs are closed class
- wolof doesn't conjugate verbs for person and tense but pronouns are
- Salishan languages may not distinguish between nouns and verbs

POS tagging is a \textbf{sequence labelling} problem since it uses context

\subsubsection{Markov Chains}%
\label{ssub:markov_chains}

We assume a \textbf{markov process}
\begin{itemize}
    \item $N$ states
    \item weighted transitions between states
    \item transition only dependent on current state
\end{itemize}

Model POS as \textbf{hidden variable} aka state, words are visible \textbf{emissions}
\begin{itemize}
    \item independence assumption allows you to factorize
    \item $P(O,Q) = P(Q_1) \prod_t P(Q_{t+1}|Q_t) \prod_t P(O_t | Q_t)$
    \item $P(the text) = P(DT) P(the | DT) P(NN | DT) P (text | NN)$
\end{itemize}

HMM for $N$ tags, $W$ words, parameters $\theta$
\begin{itemize}
    \item initial prob $Q_1$
    \item transition probs $a_{t,t+1}: Q_t \to Q_{t+1}$
    \item emission probs $b_t(O_t): Q_t \to O_t$
\end{itemize}


Choosing the most likely POS tag for each output is naive!
\begin{itemize}
    \item $P(O|\theta)$ computing likelihood of a sequence (forward/backward algorithm)
    \item $\argmax_Q P(Q, O|\theta)$ what state sequence best explains observations (Viterbi)
    \item Given an observation sequence, what is the best model? (Forward-Backward, Baum-Welch, EM)
\end{itemize}

\subsubsection{Forward Algorithm}%
\label{ssub:forward_algorithm}

$P(O|\theta) = \sum_Q P(O,Q|\theta)$

since there are $N^T$ possible paths so use DP to avoid recalculations (see Trellis in \ref{fig:forward-trellis})

\begin{figure}[h]
    \centering
    \includegraphics[width=0.8\linewidth]{comp550/forward.png}
    \caption{Trellis for Forward Algorithm}%
    \label{fig:forward-trellis}
\end{figure}

Current tag,previous words given previous tags $\alpha_i(t) = P(O_{1:t},Q_t = i | \theta)$
\begin{itemize}
    \item initial prob $\alpha_j(1) = \pi_j b_j(O_1)$
    \item recurrent sum over prev $\alpha_j(t) = \sum_i \alpha_i(t-1) a_{ij} b_j(O_t)$
    \item final $P(O|\theta) = \sum_j \alpha_j(T)$
    \item runtime $O(N^2T)$
\end{itemize}

\subsubsection{Backward Algorithm}%
\label{ssub:backward_algorithm}

Subsequent words given current tag $\beta_(t) = P(O_{t+1:T} | Q_t = i, \theta)$
\begin{itemize}
    \item excludes the current word unlike $\alpha$
    \item initial prob $\beta_j(T) = 1$
    \item recurrent sum over prev $\beta_i(t) = \sum_j a_{ij} b_j(O_{t+1}) \beta_j(t + 1) $
    \item final $P(O|\theta) = \sum_i \pi_i b_i(O_i) \beta_i(1)$
    \item runtime $O(N^2T)$
\end{itemize}

\subsubsection{Forward-Backward}%
\label{ssub:forward_backward}


Double-check forward using backward
\begin{itemize}
    \item $\alpha_i(t) \beta_i(t) = P(O, Q_t = i | \theta)$
    \item therefore $P(O | \theta) = \sum_i \alpha_i (t) \beta_i (t)$ for any $t \in [1\ldots T]$
\end{itemize}

Work in log probs for numerical stability
\begin{itemize}
    \item $\prod p_i \to \sum \log p_i$
    \item $\sum p_i \to \log \sum p_i = b + \log \sum e^{a_i - b}$ where $b = \max \log p_i$
\end{itemize}

\subsubsection{Viterbi}%
\label{ssub:viterbi}

Find most likely state sequence $Q* = \argmax_Q P(Q,O|\theta)$, like forward algorithm, but take the $\max$
\begin{itemize}
    \item initial $\delta_j(1) = \pi_j b_j(O_1) \forall j \in [1,N]$
    \item recurrence $\delta_j(t) = \max_i \delta_i (t-1) a_{ij} b_j(O_t)$
    \item final $\max_i \delta_i(T)$
    \item runtime $O(N^2T)$
\end{itemize}

Recover best $i$ for each $T$ by keeping track of $\argmax_i$ (\ref{fig:viterbi-alg})

\begin{figure}[ht]
    \centering
    \includegraphics[width=0.8\linewidth]{comp550/viterbi.png}
    \caption{Viterbi backpointers}
    \label{fig:viterbi-alg}
\end{figure}

\subsubsection{Baum-Welch}%
\label{ssub:baum_welch}

How to deal with unsupervised learning? \textbf{Hard EM} or Viterbi EM
\begin{enumerate}
    \item initialize randomly
    \item predict current state sequence using current model
    \item update current parameters based on current predictions
    \item repeat from 2
\end{enumerate}

\textbf{Baum-Welch} or soft EM uses soft predictions
\begin{itemize}
    \item expectation: get expected counts for hidden structures using $\theta_k$
    \item maximization: find $\theta_{k+1}$ to maximize likelihood on expected counts
    \item finds a local optimum in $P(O|\theta)$
\end{itemize}

Expectation of \textbf{responsabilities}: prob distribution over tags
\begin{itemize}
    \item state prob $\gamma_i(t) = P(Q_t = i|O, \theta^k) = \frac{\alpha_i(t)\beta_i(t)}{P(O|\theta^k)} $
    \item transition prob $\xi_{ij}(t) = P(Q_t = i, Q_{t+1} = j| O, \theta^k) = \frac{\alpha_i(t) a_{ij} b_j(O_{t+1})\beta_j(t+1)}{P(O|\theta^k)} $
\end{itemize}

Maximization of Soft MLE update
\begin{itemize}
    \item $\pi_i^{k+1} = \gamma_i(1)$
    \item $a_{ij}^{k+1} = \frac{\sum_t \xi_{ij}(t)}{\sum_t \gamma_i(t)} \approx \frac{count(i,j)}{count(i)} $
    \item $b_i^{k+1}(w_k) = \frac{\sum_t \gamma_{ij}(t)|_{O_t = w_k}}{\sum_t \gamma_i(t)} \approx \frac{count(w_k, i)}{count(i)}$
\end{itemize}

Practical training
\begin{itemize}
    \item stop iteration using performance on held-out set
    \item \textbf{random restarts} train different models from different inits
    \item biased initialization using some external supervised knowledge
    \item \textbf{semi-supervised} with small labelled data and large unlablled
\end{itemize}

\subsubsection{Multiword Tasks}%
\label{ssub:multiword_tasks}

Similar HMM tasks can require multiple words per tag
\begin{itemize}
    \item \textbf{Chunking} find syntactic chunks e.g. \texttt{[The chicken] [crossed] [the road]}
    \item \textbf{Named Entity Recognition} (NER) identify elements corresponding to high-level categories e.g. \texttt{[McGill University] is located in [Montreal, Canada]}
\end{itemize}

\textbf{IOB tagging} label whether word is at beginning, inside, or outside a span of tags e.g. McGill\texttt{[B-ORG]} University\texttt{[I-ORG]} is located in Montreal\texttt{[B-LOC]} Canada\texttt{[I-LOC]}

\subsection{Linear-chain CRF}%
\label{ssub:linear_chain_crf}

\subsubsection{Discriminative}%
\label{ssub:discriminative}


Shortcoming of HMMs
\begin{itemize}
    \item adding a feature requires adding an emission e.g. word position, capitalization...
\end{itemize}

\textbf{LC-CRF} linear chain conditional random field
\begin{itemize}
    \item HMM-like task-specific discriminative model $P(Y|X;\theta)$ using features, not probs
    \item $P(Y|X) = \frac{1}{Z(X)} \exp \sum_t \sum_k \theta_k f_k(y_t, y_{t-1}, x_t)$
    \item $Z(X) = \sum_y P(Y|X)$ is a normalization constant
    \item sum over all timesteps $t$, features $k$
\end{itemize}

Examples of features
\begin{itemize}
    \item HMM probs e.g. emit "the" from DT $1(y_t = DT) 1(x_t = "the")$
    \item capitalization $1(y_t = ?) 1(x_t \text{is capitalized})$
\end{itemize}


\subsubsection{Inference}%
\label{ssub:inference}

Forward algorithm computes $Z(X)$
\begin{itemize}
    \item initial prob $\alpha_j(1) = \exp \sum_k \theta_k^{init} f_k^{init}(y_1 = j, x_1)$
    \item recurrent sum over prev $\alpha_j(t) = \sum_i \alpha_i(t-1)\exp \sum_k \theta_k f_k(y_t = j, y_{t-1}, x_1) $
    \item final $Z(X) = \sum_j \alpha_j(T)$
\end{itemize}

Viterbi algorithm computes $\argmax_Y P(Y|X,\theta)$


\subsubsection{Training}%
\label{ssub:training}

Use gradient descent $\theta_{t+1} \gets \theta_t + \alpha \nabla l(\theta)$
\begin{itemize}
    \item no analytic solution
    \item $l(\theta)$ is concave
    \item can use conjugate gradient and L-BFGS
\end{itemize}

Gradient $\nabla l(\theta)$ = empirical count of features - expected feature count using current model
\begin{itemize}
    \item minimized when model matches empirical distribution
    \item regularization $\sum_k \frac{\theta_k^2}{2\sigma^2}$
    \item SGD used in practice
\end{itemize}

\subsection{Recurrent Neural Networks}%
\label{sub:recurrent_neural_networks}

\subsubsection{Neural Networks}%
\label{ssub:neural_networks}

Feed-forward NN
\begin{itemize}
    \item all computations flow forward
\end{itemize}

Time-delay NN
\begin{itemize}
    \item feed context window around word into FF
    \item requires fixed horizon
    \item sequence interaction learned indirectly, not sequence model!
\end{itemize}

\subsubsection{RNN}%
\label{ssub:rnn}

RNN: NN sequence model
\begin{itemize}
    \item modelling long-range dependencies
    \item $RNN(s_0, x_{1:n}) = s_{1:n}, y_{1:n}$
    \item $s_i = R(s_{i-1}, x_i)$
    \item $y_i = O(s_i)$
\end{itemize}

comparison to LC-CRF
\begin{itemize}
    \item LCCRF is linear, RNN is more expressive
    \item LCCRF uses feature engineering, RNN discovers features
    \item LCCRF assumes local independence, fast inference, RNN need approx (e.g. beam-search)
\end{itemize}


\subsubsection{LSTM}%
\label{ssub:lstm}

vanishing and exploding gradient
\begin{itemize}
    \item gradients wrt time 1 $ \frac{\partial L}{\partial W^1} =
        \frac{\partial L}{\partial f^N}
        \frac{\partial f^{N-1}}{\partial f^{N-2}} \ldots
        \frac{\partial f^1}{\partial W^1} $
    \item if gradient norm $<1$, it will vanish
    \item if gradient norm $>1$, it will explode to infinity
\end{itemize}

Long Short-Term Memory (Hochreiter and Schmidhuber, 1997)
\begin{itemize}
    \item explicitly model memory $C_t = f_t * C_{t-1} + i_t * \tilde C_t$
    \item forget $f_t = \sigma(W_f \cdot [h_{t-1},x_t]) $
    \item input $i_t = \sigma(W_i \cdot [h_{t-1},x_t]), \tilde C = \tanh (W_C \cdot [h_{t-1}, x_t])  $
    \item output $o_t = \sigma(W_o \cdot [h_{t-1},x_t])$
        $h_t = o_t * \tanh(C_t)$
\end{itemize}

BiLSTM, a layer for forward and backward
\begin{itemize}
    \item concatenate $h_{forward},h_{backward}$ to get output
\end{itemize}

LSTM-CRF, LCCRF layer on top of a BiLSTM
\begin{itemize}
    \item using features: output scores of LSTM $P$, transition probabilities b/w tags (learned) $A$
    \item total score $s(X,y) = \sum_i A_{y_i,y_{i+1}} + \sum_i P_{i,y_i}$
\end{itemize}

training
\begin{enumerate}
    \item forward pass for BiLSTM
    \item forward CRF to get predictions
    \item backward pass CRF to get loss
    \item backprop loss through BiLSTM
\end{enumerate}

\subsection{Pretrained LMs}%
\label{sub:pretrained_lms}

\subsubsection{Transfer Learning}%
\label{ssub:transfer_learning}

\textbf{transfer learning} using knowledge from one task to improve performance on another
\begin{itemize}
    \item don't start task from scratch, transfer knowledge of words, syntax,...
    \item use language modelling as a source task
\end{itemize}

ELMo (Peters et al, 2018) large-scale LM pretrained BiLSTM
\begin{itemize}
    \item generate contextual word embeddings
    \item $ELMo(w_k)$ = weighted sum of BiLSTM layers
\end{itemize}

\subsubsection{Transformer}%
\label{ssub:transformer}

Attention is all you need (Vaswani et al, 2017)
\begin{itemize}
    \item allow information flow between any pair of words
    \item Transformer architecture uses only attention, no recurrence
    \item $O(N^2)$ connections but better parallelism
\end{itemize}

attention, three views of a word
\begin{itemize}
    \item query: word we want to compute in the next layer
    \item key: how important the word is to another word
    \item value: value associated with the key once attention is computed
\end{itemize}


\subsubsection{Large Scale Pretrained Models}%
\label{ssub:large_scale_pretrained_models}


BERT - transformer encoder model
\begin{itemize}
    \item pretraining MLM (masked language modelling) and next sentence prediction
    \item trained on ~3B words
    \item 340M parameters
\end{itemize}

GPT-3 (OpenAI, 2020) - transformer decoder model
\begin{itemize}
    \item pretraining on language modelling
    \item trained on ~500B words
    \item up to 175B model parameters
    \item success in 0-shot and few-shot learning
\end{itemize}

Problems
\begin{itemize}
    \item generated text can be incoherent or repetitive
    \item reasoning about physics + commonsense
    \item arguments about memorization vs understanding
    \item misuse of language models (spam, fake news)
    \item fairness, bias, cost!
\end{itemize}

\section{Syntax}%
\label{sec:syntax}

\subsection{Form of Language}%
\label{sub:form_of_language}


\textbf{syntax} words arranged to form a grammatical sentence
\begin{itemize}
    \item generate all and exactly those sentences which are grammatical ( \textbf{grammaticality} )
    \item \textbf{descriptive} not \textbf{prescriptive}
\end{itemize}

\textbf{contituency} group of words that behave as a unit, e.g for noun phrases
\begin{itemize}
    \item can appear in similar syntactic envs
    \item can be replaced as a unit or rearranged
    \item can be used to answer a question
\end{itemize}
\textbf{grammatical relations} between contituents e.g.
\begin{itemize}
    \item subject
    \item (direct) object e.g. he kicked \textit{the ball}
    \item indirect object e.g. she gave \textit{him} a good beating
\end{itemize}
\textbf{subcategorization} different number and type of args mandatory for a verb or adj
\begin{itemize}
    \item (subj) relax
    \item (subj) steal (obj)
    \item (subj) want (obj / inf clause)
    \item different (from / than / to)
\end{itemize}


\subsection{Formal Grammars}%
\label{sub:format_grammars}

\textbf{formal grammar} rules that generate a set of strings to make up a language

\textbf{finite state automata} generates a regular language
\begin{itemize}
    \item correspond to \textbf{regular grammars}
    \item used for stemming, lemmatization, ...
\end{itemize}

\textbf{context-free grammars} more powerful class of grammars for NL
\begin{itemize}
    \item $N$ set of \textbf{non-terminals}
    \item $\Sigma$ of \textbf{terminals}
    \item $R$ set of rules or productions
    \item $S \in N$ start symbol
\end{itemize}

generation
\begin{itemize}
    \item \textbf{undergeneration} misses valid sentences
    \item \textbf{overgeneration} adds extra invalid sentences
    \item allows recursion
\end{itemize}

\textbf{constituency} grammars combine into bigger and bigger constituents (NP $to$ Det, N)
\begin{itemize}
    \item can be converted into dependency tree if you know the head of constituent
\end{itemize}
\textbf{dependency} grammar represent relations as directed edges
\begin{itemize}
    \item each phrase has a \textbf{head} word e.g. student \textit{studied} for the exam
    \item easier to extract relations
    \item can be converted to contituency trees if \textbf{projective} (dependecy edges don't cross, freer word order langs)
\end{itemize}

\textbf{chomsky hierarchy}
\begin{itemize}
    \item regular
    \item context-free: can describe English/German but not others
    \item context-sensitive: technically Swiss German bc of cross-serial dependencies
    \item recursively enumerable
\end{itemize}

\subsection{CYK Parsing Algorithm}%
\label{sub:cyk_parsing_algorithm}

\subsubsection{CYK Parsing}%
\label{ssub:cyk_parsing}

\textbf{parsing} given input sentence and CFG, recover all possible parse trees
\begin{itemize}
    \item top-down: start at $S$ and expand (e.g. Earley)
    \item bottom-up: start from input words (e.g. CYK, shift-reduce)
\end{itemize}

\textbf{CYK} (Cocke-Younger-Kasami) bottom-up dynamic programming
\begin{itemize}
    \item convert CFG into Chomsky Normal Form
    \item setup table of contituents
    \item fill in table
    \item use table
\end{itemize}

\textbf{Chomsky Normal Form}: $A \to B,C$ non-terminal, $A \to s$ terminal
\begin{itemize}
    \item reduce non-terminals to 2 $A \to B,C,D$ converts $A \to X1,D$ and$X1 \to B,C $
    \item split non-terminal,terminal $A \to s,B$ converts $A \to X2,B$ and $X2 \to s$
    \item remove single non-terminals $A \to B, B \to \ldots$ converts $A \to \ldots$
\end{itemize}

constituent parse table (see \ref{fig:parse-table})
\begin{itemize}
    \item sentence $w[0], w[1] \ldots w[N-1]$
    \item cell $i,j$ corresponds to span $w[i:j+1]$
    \item cell lists non-terminals that can span those words
\end{itemize}

\begin{figure}[ht]
    \centering
    \includegraphics[width=0.7\linewidth]{comp550/parse-table.png}
    \caption{contituent parse table}%
    \label{fig:parse-table}
\end{figure}

build parse table
\begin{itemize}
    \item base: for each word $i$ put in $c[i:i+1]$ non-terminal that generates the word
    \item recursive: check all break points $m \in [i,j]$ and check for rule $A[i,j] \to B[i:m],C[m:j]$
    \item fill in bottom-up, left-to-right
\end{itemize}

use table
\begin{itemize}
    \item use cell $c[0:N]$ and follow back-pointers
\end{itemize}

\subsubsection{Probabilistic CYK Parsing}%
\label{ssub:probabilistic_cyk_parsing}

\textbf{probabilistic CYK} associate each rule with a probability, use product of probabilities for rules
\begin{itemize}
    \item for each non-terminal $A, \sum_B Pr(A \to B) = 1$
    \item combining rules for tree  $t, Pr(t) = \prod_i Pr(A_i \to B_i)$ with rules $A_i \to B_i$
\end{itemize}

\textbf{probabilistic parsing} recovers best parse $\argmax_t Pr(t)$
\begin{itemize}
    \item naive: run CYK and take argmax
    \item idea: keep track of probability in CYK and always choose highest probability parse
\end{itemize}

\textbf{probabilistic CYK} keeps track of probability and only most likely for each tree type
\begin{itemize}
    \item table[2, 4, NP] = 0.215 (table[2,3,Det], table[3,4,N])
    \item table[3, 4, NP] = 0.022
    \item table[3, 4, N] = 0.04
\end{itemize}

\textbf{Vanilla PCFG} learn probabilities using MLE on corpus (e.g. WSJ)
\begin{itemize}
    \item not enough context e.g. "I" "me" are NP but are object or subject
    \item rules are too sparse e.g $VP \to VBD$ and $VP \to VBD,PP$ are separate not factorizable
    \item strong \textit{vertical} assumption, contituent independent of other parts of tree
    \item weak \textit{horization} assumption, tries to model all combinations separately
\end{itemize}

\subsubsection{Markovization}%
\label{ssub:markovization}

Improved PCFG by markovization: better context annotation and factorization
\begin{itemize}
    \item \textbf{vertical} annotate with $n$-th parent $NP^S$ likey subject, $NP^{VP}$ likely object
    \item \textbf{horizontal} factorize large rules like $n$-gram $Pr(VP \to START,AdvP,VBD \ldots) = Pr(VP \to START,AdvP) * Pr(VP \to AdvP,VBD) \ldots$ and learn factors separately
\end{itemize}

WSJ results by Klein and Manning (2003) show large improvement $71.3 \to 79.7$


\section{Semantics}%
\label{sec:semantics}

\subsection{Meaning of Language}%
\label{sub:meaning}

\textbf{semantics} study of meaning in language
\begin{itemize}
    \item \textbf{extensional} picks out word's \textbf{referents} in the real world
    \item \textbf{intensional} defines a word in terms of other words e.g. dictionary
\end{itemize}

Frege (1892) differentiates sense and reference
\begin{itemize}
    \item \textbf{sense} of a term is its meaning, whereas
    \item \textbf{reference} is what it points to in the real world
\end{itemize}

\subsection{Lexical Semantics}%
\label{sub:lexical_semantics}

\textbf{lexical semantics} the meaning of words, with relations
\textbf{hyponym / hypernym} hypo "is a" hyper
\begin{itemize}
    \item class: monkey / mammal
    \item instance: Montreal / city
\end{itemize}
\textbf{synonym} same meaning
\textbf{antonym} opposite meaning
\textbf{homonym} same form, different unrelated meaning
\begin{itemize}
    \item homophone: same sound
    \item homograph: same written form
\end{itemize}
\textbf{polysem} multiple related meanings
\textbf{metonym} substituting one entity for related one (e.g. order a \textit{dish} from the menu)
\begin{itemize}
    \item synecdoche: whole-part relation (e.g. all \textit{hands} on deck)
\end{itemize}
\textbf{meronym / holonym} mero "is part of" holo
\begin{itemize}
    \item groups and members (student / class)
    \item whole and part (wheel / car)
    \item whole and substance (wood / chair)
\end{itemize}


WordNet (Miller et al, 1990) lexical resource organized by \textbf{synsets}
\begin{itemize}
    \item hierarchies based on lexical relations
\end{itemize}

\subsubsection{Word Sense Disambiguation}%
\label{ssub:word_sense_disambiguation}

\textbf{word sense disambiguation} figure out which sense a word is expressing using contextual words

\textbf{Lesk's algorithm} (1986) heuristic approach
\begin{enumerate}
    \item construct $B$ bag-of-words rep for context
    \item calculate $overlap(B, signature(s_i)$ for each candidate sense $s_i$
    \item choose sense with highest overlap
\end{enumerate}

\textbf{Yarowsky's algorithm} (1990) unsupervised (or minimal) based on bootstrapping
\begin{enumerate}
    \item choose word to disambiguate (e.g. plant)
    \item find two senses and label with heuristic (e.g. plant \textit{life} vs \textit{manufacturing}) creating \textit{seed set}
    \item repeat iteration
        \begin{itemize}
            \item learn a supervised model on seed set
            \item predict labels of whole dataset
            \item keep highly confident labels as next seed set
        \end{itemize}
\end{enumerate}

\subsubsection{Hearst Patterns}%
\label{ssub:hearst_patterns}

detect lexical semantic relationships in text since they tend to occur in certain \textbf{lexico-syntactic} patterns

Hearst (1992) patterns for hyper-hypo
\begin{itemize}
    \item NP such as {NP} {and|or} NP
    \item NP {,} including {NP,}
    \item NP {,} especially {NP,}
\end{itemize}

Find new patterns by bootstrapping with known pairs $\to$ patterns $\to$ new pairs

\subsection{Distributional Semantics}%
\label{sub:distributional_semantics}

some semantic rels, e.g. synonyms, do not appear together but rather occur in the same context

``You shall know a word by the company it keeps'' (Firth, 1957)

\subsubsection{Word Vectors}%
\label{ssub:word_vectors}

\textbf{Count-based} word vectors
\begin{itemize}
    \item for some word $i$, keep count of words that appear within a window of $n$ words
    \item generate term-context matrix such as \ref{fig:term-context}
    \item compute cosine sim between word vectors to get similarity
\end{itemize}

\begin{figure}[h]
    \centering
    \includegraphics[width=0.6\linewidth]{comp550/term-context.png}
    \caption{term-context matrix}%
    \label{fig:term-context}
\end{figure}

vector similarity measures \textit{relatedness}
\begin{itemize}
    \item synonyms and antonyms are hard to distinguish by context
    \item \textbf{similarity} synonymy, hypernymy, hyponymy
    \item \textbf{relatedness} association, includes antonyms
\end{itemize}

evaluating word vectors is non-trivial
\begin{itemize}
    \item compare to gold standard e.g. WS-353 (Finkelstein et al, 2002)
\end{itemize}

\subsubsection{Better Word Vectors}%
\label{ssub:better_word_vectors}

instead of raw counts use \textbf{Pointwise MI}
\begin{itemize}
    \item $\log \frac{P(w_1,w_2)}{P(w_1)P(w_2)}$
    \item normalizes chance co-occurence
    \item PMI can measure positive correlation $>0$ or negative $<0$
    \item \textbf{Positive PMI} sets $PMI < 0 \to PMI = 0$
\end{itemize}

\subsubsection{Singular Value Decomposition}%
\label{ssub:singular_value_decomposition}

term-context matrices are sparse, compress into smaller dense

\textbf{Singular Value Decomposition} matrix factorization $X = W \times \Sigma \times C^T$
\begin{itemize}
    \item new word vectors $W \in R^{|V| \times m}$
    \item $\Sigma$ has singular values of $X$, arranged highest to lowest
\end{itemize}

\textbf{truncated SVD} take top $k$ values in $\Sigma$
\begin{itemize}
    \item use $W_k \times \Sigma_k$ as word vectors
    \item corresponds to finding smallest reconstruction error
    \item same as PCA, projecting to top $k$ components
\end{itemize}

\subsubsection{Word2Vec}%
\label{ssub:word2vec}

\textbf{word2vec} (Mikolov et al, 2013) trains NNs to predict words in context
\begin{itemize}
    \item vector representations known as \textbf{word embeddings}
    \item CBOW: use context words vec $c_j$ to predict target word vec $v_j$
    \item Skip-gram: use target to predict context $P(w_k | w_j) = \frac{\exp{c_k \cdot v_j}}{\sum_{i \in |V|} \exp{c_i \cdot v_j}}$
    \item \textbf{negative sampling} to approximate denominator, separate true context from fake context
\end{itemize}

hyperparameters are key to performance
- weighting important of context words
- negative sampling $p^{3/4}(w)$ is better than $p(w)$
- similar changes improve other methods like SVD

widely used in NLP
- cheap, easy, large pretrained
- doesn't always work for task-specific

\subsection{Compositional Semantics}%
\label{sub:compositional_semantics}

\subsubsection{Compositionality}%
\label{ssub:compositionality}

\textbf{compositionality} (c) the meaning of a phrase depends on the meaning of its parts
\begin{itemize}
    \item \textbf{non-c} e.g. idioms ``kick the bucket''
    \item \textbf{co-c} meaning depends on composed words e.g. \textit{red} hair vs \textit{red} wine
\end{itemize}

\textbf{c semantics} derivate a good meaning of a representation from its parts
\begin{itemize}
    \item assert a falsifiable proposition
    \item convey information about the world
    \item query about the world
\end{itemize}

\subsubsection{Semantic Inference}%
\label{ssub:semantic_inference}

Montague (1970) used logical formalism to represent meaning

\textbf{semantic inference} make explicit something that is implicit in language (Blackburn and Bos, 2003)

First-order logic
\begin{itemize}
    \item \textbf{domain of discourse} a set of entities
    \item \textbf{variables} potential elements of domain $x$
    \item \textbf{predicates} maps elements to truth value $inCourse(x,y)$
    \item \textbf{functions} maps elements to elements $instructorOf(x) \to y$
    \item \textbf{logical connectives} $\neg, \vee, \wedge, \to \ldots$
    \item \textbf{quantifiers} $\exists, \forall$
\end{itemize}
interpretation or \textbf{model} of a FOL
\begin{itemize}
    \item domain of discourse $D$
    \item mapping of functions to $D$
    \item mapping of predicates to True or False
\end{itemize}

\subsubsection{Lambda Calculus}%
\label{ssub:lambda_calculus}

we need a computation procedure to convert from NL to formal logic

\textbf{lambda calculus}
\begin{itemize}
    \item variable $x$
    \item $\lambda x.t$ where $t$ is a lambda term
    \item $ts$ where both are lambda terms
    \item \textbf{beta reduction} function application of $(\lambda x.t)s$
    \item note that it is \textbf{left associative} $abcd = ((ab)c)d$
\end{itemize}

construct contituents with lambda calculus
\begin{itemize}
    \item ``disdained'' $\lambda x.\lambda y. disdained(y,x)$
    \item ``disdained catnip'' $\lambda y. disdained(y, catnip)$
    \item ``Whiskers disdained catnip'' $(\lambda y. disdained(y, catnip))Whiskers$
\end{itemize}

\textbf{syntax-driven sem composition}: augment CFG with lambda expression
\begin{itemize}
    \item syntax $A \to a_1, \ldots a_n$ semantic attachment $\{f(\alpha_1.sem, \ldots \alpha_n.sem)\}$
    \item common noun $N \to student, \{ \lambda x.Student(x)\}$
    \item proper noun $PN \to COMP550, \{\lambda x.x(COMP550)\}$ \textbf{type-raised} to be same as common
    \item intransitive verb $V \to rules, \{ \lambda x. \exists e Rules(e) \wedge Ruler(e,x) \}$ \\
        \textbf{neo-davidsonian} event semantics $Rules(x)$ uses reified event $e$
    \item composition is function application $S \to NP, VP, \{NP.sem(VP.sem)\} $
\end{itemize}

\subsubsection{Quantification}%
\label{sub:quantification}

quantifiers
\begin{itemize}
    \item universal \textit{all}, $\forall x \ldots \to \ldots$
    \item existential \textit{a}, $\exists x \ldots \wedge \ldots$
    \item definite \textit{the} (Russell, 1905)
        \begin{itemize}
            \item existence $\exists x.Student(x)$
            \item uniqueness (at most one) $\wedge \forall y.Student(y) \to x = y$
            \item predicate $\wedge Smart(x)$
        \end{itemize}
\end{itemize}

more semantic attachments
\begin{itemize}
    \item universal $\lambda P. \lambda Q. \forall x. P(x) \to Q(x)$
    \item existential $\lambda P. \lambda Q. \exists x. P(x) \wedge Q(x)$
    \item adjectives $ \lambda N.\lambda x. Smart(x) \wedge N(x)$
\end{itemize}

\subsubsection{Quantifier Scope Ambiguity}%
\label{ssub:quantifier_scope_ambiguity}

\textbf{scope ambiguity} when multiple quantifier have multiple meanings e.g. ``\textit{Every} student took \textit{a} course''
\begin{itemize}
    \item different course $\forall x.Student(x) \to (\exists y.Course(y) \ldots)$
    \item same course $\exists y.Course(y) \wedge (\forall x.Student(x) \ldots)$
\end{itemize}

\textbf{underspecification} derive representation represents \textit{all possible} meanings

\textbf{Cooper storage} (1988) associates store with each FOL
\begin{align*}
    \exists e.took(e) &\wedge taker(e, \color{red} s_1 \color{black} ) \wedge takee(e,\color{blue}{s_2} \color{black}) \\
                      &(\lambda Q. \forall x.Student(x) \to Q(x), \color{red} 1 \color{black}), \\
                      &(\lambda Q. \exists y.Course(y) \wedge Q(y), \color{blue} 2 \color{black})
\end{align*}

recover reading from cooper storage
\begin{enumerate}
    \item select quantifier order (e.g. $s_1$ first)
    \item do lambda abstraction for each quantifier and apply function (beta-reduce)
\end{enumerate}

\begin{align*}
    \lambda Q. &\forall x.Student(x) \to Q(x) \\
               & \color{red} \lambda s_1 \color{black} \exists e.took(e) \wedge taker(e, \color{red} s_1 \color{black} ) \wedge takee(e,\color{blue}{s_2} \color{black}) \\
                      &(\lambda Q. \exists y.Course(y) \wedge Q(y), \color{blue} 2 \color{black})
\end{align*}


semantic attachment quantifier composition
\begin{itemize}
    \item compose by putting in storage
    \item $NP \to Det, N$ gives $\{ \lambda u.u(\color{red} s_i \color{black}), \text{storage:} (Det.sem(N.sem), \color{red} i \color{black}) \}$
\end{itemize}


\section{Discourse}%
\label{sec:discourse}

\subsection{Language Communication}%
\label{sub:phrases_of_language}

language is a \textit{discourse}
\begin{itemize}
    \item \textbf{monologue} one-directional communication
    \item \textbf{dialogue} multi-participant communication
\end{itemize}

sequential sentences must ``make sense''
\begin{itemize}
    \item \textbf{coherence} is about logical sense
    \item \textbf{cohesion} is about linguistic devices to flow better
        \begin{itemize}
            \item \textbf{lexical chain} of semantically similar words e.g.  The \textit{government} proposes rules. \textit{Citizens} are protesting.
            \item \textbf{anaphoric devices} like \textbf{coreference chains} when one phrase meaning depends on another e.g.  \textit{The rules} are long. \textit{These regulations} \ldots
            \item \textbf{discourse markers} of cue words e.g. The dog is big. Dogs can \textit{also} be small.
        \end{itemize}
\end{itemize}

\subsection{Coreference Resolution}%
\label{sub:coreference_resolution}

\subsubsection{Corefence and Anaphora}%
\label{ssub:corefence_and_anaphora}

\textbf{reference} relating \textit{mentions} to \textit{referents} in the real world
\begin{itemize}
    \item proper names \textit{Montreal}
    \item pronouns \textit{he, she}
    \item noun phrases
        \begin{itemize}
            \item indefinite \textit{a} deer
            \item definite \textit{the} cat
        \end{itemize}
    \item demonstrative: points to something \textit{this} hotdog
\end{itemize}

\textbf{coreference} when two mentions point to the same object
\begin{itemize}
    \item \textbf{anaphor} points to previous Maru is a cat. \textit{He} is grumpy
    \item \textbf{cataphor} points to following When \textit{he} is grumpy, Maru doesn't eat
\end{itemize}

\textbf{zero anaphora} in other \textbf{pro-drop} languages allows omitting pronouns
\begin{itemize}
    \item e.g. Spanish can figure out pronoun using verb \textit{hablo} Español
    \item e.g. Japanese requires context \textit{Ai shi} (love) te ru (prog pres) = I love you
\end{itemize}

noun phrase coreference
\begin{itemize}
    \item \textbf{bridging} infer ref from previous mentions ``I like my office. \textit{The windows} are large''
    \item \textbf{pleonastic} are non-referential pronounce ``\textit{It} is raining''
    \item \textbf{clefting} puts focus on something (sorta referential) ``\textit{It} is my cat that is grumpy''
\end{itemize}

coreference beyond entities and noun phrases
\begin{itemize}
    \item \textbf{event} that corefer or happen in same time / place \\ ``I \textit{bought} my cat. The \textit{adoption} was quick''
    \item \textbf{abstraction} shell noun (this) represents a whole thing  \\ ``Grumpiest cats are the best. \textit{This fact} is true''
    \item \textbf{cross document} alignment of events and entities from multiple sources
\end{itemize}

\subsubsection{Hobbs Algorithm}%
\label{ssub:hobbs_algorithm}

cues for anaphora resolution
\begin{itemize}
    \item number and gender
    \item recency
    \item syntactic info e.g. \textbf{binding theory} (Chomsky, 1981)
        \begin{itemize}
            \item ``the students taught \textit{themselves}''
            \item reflexives are bound by a subject in a certain way
            \item personal pronouns are not
        \end{itemize}
\end{itemize}

Hobbs (1978) traversal algorithm using constituent parse tree and morph analysis of number and gender
\begin{enumerate}
    \item search the sentence right-to-left
    \item use heuristics and check if NP found match in number and gender
    \item if no antecedent found, search previous sentences left-to-right
\end{enumerate}


\subsubsection{Coreference ML}%
\label{ssub:coreference_ml}

coreference subtasks
\begin{itemize}
    \item \textbf{mention detection} decide which text are mentions and anaphoric
    \item \textbf{coreference resolution} determine coref links in passage
\end{itemize}

Soon et al (2001) defined 12 features and used decision tree
\begin{itemize}
    \item string overlap
    \item pronoun
    \item wordnet class
    \item number agreement
    \item gender agreement
    \item distance (recency)
\end{itemize}

Ng and Cardie (2002) extended the feature set, Durrett and Klein (2013) incorporated even more features into log-linear model


Lee et al (2017) supervised end-to-end neural coref
\begin{itemize}
    \item $s_m(i)$ mention score of $i$
    \item $s_a(i,j)$ score of $j$ antecedent of $i$
    \item score all possible pairs, prune low-prob mentions
\end{itemize}


\subsection{Coherence Modelling}%
\label{sub:coherence_modelling}

theories of coherence
\begin{itemize}
    \item discourse relations b/w text spans e.g. \textbf{rhetorical structure}
    \item \textbf{local coherence} links between text spans
    \item \textbf{global coherence} structure of entire discourse
\end{itemize}

\subsubsection{Rhetorical Strucutre Theory}%
\label{ssub:rhetorical_strucutre_theory}

Mann and Thomson (1988) describe discourse structure as tree e.g. \ref{fig:rst-tree}
\begin{enumerate}
    \item segement text into \textbf{EDU} elementary discourse units
    \item relate spans according to \textbf{rhetorical relations}
\end{enumerate}

rhetorical relations (nucleus-sattelite)
\begin{itemize}
    \item elaboration
    \item attribution (gives source of information)
    \item contrast
    \item list (without explicit contrast or comparison)
    \item background info
\end{itemize}

relation components
\begin{itemize}
    \item \textbf{nucleus-sattelite} for asymmetric
    \item \textbf{nucleus-nucleus} for symmetric
\end{itemize}

\begin{figure}[ht]
    \centering
    \includegraphics[width=0.8\linewidth]{comp550/rst-tree.png}
    \caption{RST Tree from Joy et al. (2013)}%
    \label{fig:rst-tree}
\end{figure}

applications using RST Tree features
\begin{itemize}
    \item essay grading
    \item automatic summarization
\end{itemize}

bootstrapping
\begin{itemize}
    \item gather discourse cues ``consequently'' $\to RESULT$
    \item supervised learn the discourse value
\end{itemize}


\subsubsection{Local Coherence Modelling}%
\label{ssub:local_coherence_modelling}

LCM (Barzilay and Lapata, 2005) looks at local cohesives in adjacent sentences

Generally, entity mentions follow patterns
\begin{itemize}
    \item first mention is often subject ``Maru is \ldots''
    \item mentions are clustered together ``\ldots Maru is grumpy. He likes cat food. \ldots (no more mentions)''
\end{itemize}

\textbf{entity grid} (Barzilay and Lapata, 2008)
\begin{itemize}
    \item plots entity mentions and syntactic role (see \ref{fig:entity-grid})
    \item extract features of document using relative freq of entity mention transition e.g. prob of subject $\to$ subject
    \item use to model ordering of a document, summary coherence, readability
\end{itemize}

\begin{figure}[ht]
    \centering
    \includegraphics[width=0.5\linewidth]{comp550/entity-grid.png}
    \caption{Entity grid across 6 sentences (Bazilay and Lapata, 2008) with grammatical roles subject (s), object (o), or neither (x)}%
    \label{fig:entity-grid}
\end{figure}

entity grid extensions
\begin{itemize}
    \item combining global and local coherence (Elsner et al, 2007)
    \item other languages (Cheung and Penn, 2010)
    \item neural coherence (Nguyen and Joty, 2017)
    \item self-supervised (Xu et al, 2019)
        \begin{itemize}
            \item encode consecutive sentences $s, t$
            \item learn score using $s,t$ features
            \item SSL with NCE
        \end{itemize}
\end{itemize}

\end{document}
