\documentclass[]{article}
\usepackage{etex}
\usepackage[margin = 1.5in]{geometry}
\setlength{\parindent}{0in}
\usepackage{amsmath}
\usepackage{amsfonts}
\usepackage{amssymb}
\usepackage{amsthm}
\usepackage{listings}
\usepackage{color}
\usepackage{mathtools}
\usepackage{multicol}
\usepackage[lined]{algorithm2e}
\usepackage{float}
\usepackage[T1]{fontenc}
\usepackage{ae,aecompl}
\usepackage[pdftex,
  pdfauthor={Michael Noukhovitch},
  pdftitle={},
  pdfsubject={Lecture notes from },
  pdfproducer={LaTeX},
  pdfcreator={pdflatex}]{hyperref}

\usepackage{cleveref}
\usepackage{enumitem}

\definecolor{dkgreen}{rgb}{0,0.6,0}
\definecolor{gray}{rgb}{0.5,0.5,0.5}
\definecolor{mauve}{rgb}{0.58,0,0.82}

\lstset{
  language=C,
  aboveskip=3mm,
  belowskip=3mm,
  showstringspaces=false,
  columns=flexible,
  basicstyle={\small\ttfamily},
  numbers=none,
  numberstyle=\tiny\color{gray},
  keywordstyle=\color{blue},
  commentstyle=\color{dkgreen},
  stringstyle=\color{mauve},
  breaklines=true,
  breakatwhitespace=true,
  tabsize=4
}

\theoremstyle{definition}
\newtheorem*{defn}{Definition}
\newtheorem{ex}{Example}[section]
\newtheorem*{theorem}{Theorem}

\setlength{\marginparwidth}{1.5in}
\setlength{\algomargin}{0.75em}

\DeclarePairedDelimiter{\set}{\lbrace}{\rbrace}

\definecolor{darkish-blue}{RGB}{25,103,185}

\usepackage{hyperref}
\hypersetup{
    colorlinks,
    citecolor=darkish-blue,
    filecolor=darkish-blue,
    linkcolor=darkish-blue,
    urlcolor=darkish-blue
}
\newcommand{\lecture}[1]{\marginpar{{\footnotesize $\leftarrow$ \underline{#1}}}}

\makeatletter
\def\blfootnote{\gdef\@thefnmark{}\@footnotetext}
\makeatother

\begin{document}
	\let\ref\Cref

	\title{\bf{Natural Language Processing}}
	\date{Fall 2020, McGill\\ \center Notes written from Jackie Cheung's lectures}
	\author{Michael Noukhovitch}

	\maketitle
	\newpage
	\tableofcontents
	\newpage

    \section{Introduction}%
    \label{sec:introduction}

    \subsection{Overview}%
    \label{sub:overview}

    \textbf{language} is a form of communication
    \begin{itemize}
        \item \textit{arbitrary} pairing between form and meaning
        \item very expressive and productive
        \item nearly universal
        \item uniquely human*
    \end{itemize}

    \textbf{computational linguistics} modelling natural language with computational models
    \begin{itemize}
        \item acoustic signals
        \item NL understanding (comprehension)
        \item NL generation (production)
    \end{itemize}

    goals of the field
    \begin{itemize}
        \item practical technologies (NLP)
        \item understanding how language works (CL)
    \end{itemize}

    models and techniques
    \begin{itemize}
        \item gathering data
        \item evaluation
        \item statistical methods (ML)
        \item rule-based systems
    \end{itemize}

    some example problems
    \begin{itemize}
        \item is language an instinct? (Chomsky)
        \item language processing to understand meaning of sentence
        \item can we learn mathematical properties of language
    \end{itemize}

    types of language
    \begin{itemize}
        \item \textbf{text} an idealization of spoken language
            \begin{itemize}
                \item luckily English is similar between writing and speaking, and there is lots of data on it
                \item older work used ``clean'' language but recent work ventures into messy data (e.g. Twitter)
            \end{itemize}
        \item \textbf{speech} is much messier
            \begin{itemize}
                \item automatic speech recognition (ASR)
                \item text-to-speech generation (TTS)
            \end{itemize}
    \end{itemize}

    \subsection{Domains of Language}%
    \label{sub:domains_of_language}

    \textbf{phonetics} study of speech sounds
    \begin{itemize}
        \item articulation, transmission
        \item how each sound is made in the mouth
    \end{itemize}
    \textbf{phonology} rules that govern sound patterns
    \begin{itemize}
        \item how the sounds are organized
        \item ``p'' in peach and speach are the same phoneme but phonetically distinct (aspiration)
    \end{itemize}
    \textbf{morphology} word formation and meaning
    \begin{itemize}
        \item anti-dis-establish-ment-arian-ism
    \end{itemize}
    \textbf{syntax} structure of language
    \begin{itemize}
        \item ``I a woman saw park in the'' is \textbf{ungrammatical}
        \item \textbf{ambiguity} different possible meaning for the same phrase
    \end{itemize}
    \textbf{semantics} meaning of language
    \begin{itemize}
        \item ``Ross wants to marry \textbf{a} Swedish woman''
    \end{itemize}
    \textbf{pragmatics} meaning of language in context
    \begin{itemize}
        \item different from literal meaning
        \item \textbf{deixis} interpretation that relies on extra-linguistic context
        \item ``dessert would be delicious''
    \end{itemize}
    \textbf{discourse} structure of larger spans of language
    \begin{itemize}
        \item do large spans of text form a coherent story
    \end{itemize}

    \subsection{Technology}%
    \label{sub:technology}

    combination of hand-crafted knowledge and ML on data
    \begin{itemize}
        \item rule-based systems
        \item machine learning
        \item knowledge representation
    \end{itemize}


\end{document}
