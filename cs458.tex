\documentclass[]{article}
\usepackage{etex}
\usepackage[margin = 1.5in]{geometry}
\setlength{\parindent}{0in}
\usepackage{amsmath}
\usepackage{amsfonts}
\usepackage{amssymb}
\usepackage{amsthm}
\usepackage{listings}
\usepackage{color}
\usepackage{mathtools}
\usepackage{multicol}
\usepackage{pgfplots}
\usepackage{qtree}
\usepackage{xytree}
\usepackage[lined]{algorithm2e}
\usepackage{float}
\usepackage[T1]{fontenc}
\usepackage{ae,aecompl}
\usepackage[pdftex,
  pdfauthor={Michael Noukhovitch},
  pdftitle={CS 458: Computer Security and Privacy},
  pdfsubject={Lecture notes from CS 458 at the University of Waterloo},
  pdfproducer={LaTeX},
  pdfcreator={pdflatex}]{hyperref}

\usepackage{cleveref}
\usepackage{enumitem}

\definecolor{dkgreen}{rgb}{0,0.6,0}
\definecolor{gray}{rgb}{0.5,0.5,0.5}
\definecolor{mauve}{rgb}{0.58,0,0.82}

\lstset{
  language=C,
  aboveskip=3mm,
  belowskip=3mm,
  showstringspaces=false,
  columns=flexible,
  basicstyle={\small\ttfamily},
  numbers=none,
  numberstyle=\tiny\color{gray},
  keywordstyle=\color{blue},
  commentstyle=\color{dkgreen},
  stringstyle=\color{mauve},
  breaklines=true,
  breakatwhitespace=true,
  tabsize=4
}

\theoremstyle{definition}
\newtheorem*{defn}{Definition}
\newtheorem{ex}{Example}[section]
\newtheorem*{theorem}{Theorem}

\setlength{\marginparwidth}{1.5in}
\setlength{\algomargin}{0.75em}

\DeclarePairedDelimiter{\set}{\lbrace}{\rbrace}

\definecolor{darkish-blue}{RGB}{25,103,185}

\usepackage{hyperref}
\hypersetup{
    colorlinks,
    citecolor=darkish-blue,
    filecolor=darkish-blue,
    linkcolor=darkish-blue,
    urlcolor=darkish-blue
}
\newcommand{\lecture}[1]{\marginpar{{\footnotesize $\leftarrow$ \underline{#1}}}}

\makeatletter
\def\blfootnote{\gdef\@thefnmark{}\@footnotetext}
\makeatother

\begin{document}
	\let\ref\Cref

	\title{\bf{CS 458: Computer Security and Privacy}}
	\date{Spring 2016, University of Waterloo \\ \center Notes written from Erinn Atawater's lectures.}
	\author{Michael Noukhovitch}

	\maketitle
	\newpage
	\tableofcontents
	\newpage

	\section{Introduction}
	\subsection{Security}
	Security can be defined as:
	\begin{description}
		\item[confidentiality] access to systems is limited to authorized
		\item[integrity] getting the correct data
		\item[availability] system is there when you want it
	\end{description}
	\subsection{Privacy}
	There are many definitions but we will stick to \textbf{informational self-determination}, where you control the information about you

	\subsection{Terminology}
	\begin{description}
		\item[assets] things we want to protect
		\item[vulnerabilities] weaknesses in a system that can be exploited
		\item[threats] loss or harm that may befall a system
			\begin{itemize}
				\item interception
				\item interruption
				\item modification
				\item fabrication
			\end{itemize}
		\item[threat model] set of threats to defend against (who/what)
		\item[attack] an action which exploits a vulnerability to execute a threat
		\item[control] removing or reducing a vulnerability
	\end{description}

	\subsection{Types of Defence}
	Defend against an attack:
	\begin{itemize}
		\item \textbf{prevent} stop the attack from happening
		\item \textbf{deter} make the attack more difficult
		\item \textbf{deflect} make it less attractive for attacker
		\item \textbf{recover} mitigate effects of the attack
	\end{itemize}

	Make sure that defence is correct with principles:
	\begin{itemize}
		\item \textbf{easiest penetration} system is only as strong as weakest link
		\item \textbf{adequate protection} don't spend more on defence than the value of the system
	\end{itemize}

	\subsection{Methods of Defence}
	\begin{itemize}
		\item Software controls: passwords, virus scanner \dots
		\item Hardware controls: fingerprint reader, smart token \dots
		\item Physical controls: locks, guards, backups \dots
		\item Policies: teaching employees, password changing rules
	\end{itemize}


	\section{Program Security}
	\subsection{Flaws, faults, and failures}
	\subsubsection{Defintions}
	\begin{description}
		\item[flaw] problem with a program
		\item[fault] a potential error inside the logic
		\item[failure] an actual error visible by the user
	\end{description}
	\subsubsection{Unexpected Behaviour}

\end{document}
