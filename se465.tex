\documentclass[]{article}
\usepackage{etex}
\usepackage[margin = 1.5in]{geometry}
\setlength{\parindent}{0in}
\usepackage{amsmath}
\usepackage{amsfonts}
\usepackage{amssymb}
\usepackage{amsthm}
\usepackage{listings}
\usepackage{color}
\usepackage{enumitem}
\usepackage{mathtools}
\usepackage{pgfplots}
\usepackage[lined]{algorithm2e}
\usepackage{qtree}
\usepackage{xytree}
\usepackage{float}
\usepackage[T1]{fontenc}
\usepackage{ae,aecompl}
\usepackage{multicol}
\usepackage[pdftex,
  pdfauthor={Michael Noukhovitch},
  pdftitle={SE 465: Testing},
  pdfsubject={Lecture notes from SE 465 at the University of Waterloo},
  pdfproducer={LaTeX},
  pdfcreator={pdflatex}]{hyperref}

\usepackage{cleveref}

\definecolor{dkgreen}{rgb}{0,0.6,0}
\definecolor{gray}{rgb}{0.5,0.5,0.5}
\definecolor{mauve}{rgb}{0.58,0,0.82}

\lstset{
  language=Java,
  aboveskip=3mm,
  belowskip=3mm,
  showstringspaces=false,
  columns=flexible,
  basicstyle={\small\ttfamily},
  numbers=none,
  numberstyle=\tiny\color{gray},
  keywordstyle=\color{blue},
  commentstyle=\color{dkgreen},
  stringstyle=\color{mauve},
  breaklines=true,
  breakatwhitespace=true,
  tabsize=4
}

\lstdefinelanguage{JavaScript}{
	keywords={break, case, catch, continue, debugger, default, delete, do, else, finally, for, function, if, in, instanceof, new, return, switch, this, throw, try, typeof, var, void, while, with},
	morecomment=[l]{//},
	morecomment=[s]{/*}{*/},
	morestring=[b]',
	morestring=[b]``,
	sensitive=true
}


\theoremstyle{definition}
\newtheorem*{defn}{Definition}
\newtheorem{ex}{Example}[section]
\newtheorem*{theorem}{Theorem}

\setlength{\marginparwidth}{1.5in}
\setlength{\algomargin}{0.75em}

\DeclarePairedDelimiter{\set}{\lbrace}{\rbrace}

\definecolor{darkish-blue}{RGB}{25,103,185}

\usepackage{hyperref}
\hypersetup{
    colorlinks,
    citecolor=darkish-blue,
    filecolor=darkish-blue,
    linkcolor=darkish-blue,
    urlcolor=darkish-blue
}
\newcommand{\lecture}[1]{\marginpar{{\footnotesize $\leftarrow$ \underline{#1}}}}

\makeatletter
\def\blfootnote{\gdef\@thefnmark{}\@footnotetext}
\makeatother

\begin{document}
	\let\ref\Cref

	\title{\bf{SE 465: Testing}}
	\date{Winter 2015, University of Waterloo \\ \center Notes written from Patrick Lam's lectures.}
	\author{Michael Noukhovitch}

	\maketitle
	\newpage
	\tableofcontents
	\newpage

	\section{Introduction}
		\subsection{Types of Problems}
			\begin{itemize}
				\item \textbf{fault}: static defect in the software
				\begin{itemize}
					\item \textbf{design fault}
					\item \textbf{mechanical fault}
				\end{itemize}				 
				\item \textbf{error}: have incorrect state
				\item \textbf{failure}: external incorrect behavior
			\end{itemize}
			\begin{ex}
				Faults
				\begin{lstlisting}{language=Java}
	static public int findLast (int[] x, int y) {
		for (int i=x.length-1; i>0; --i){
	    	if (x[i] == y){
	        	return i;
	        }
	   	}
	   	return -1;
	}			
				\end{lstlisting}
				
				\textbf{fault}: should be \lstinline|i >= 0| \\
				no \textbf{fault} input: \lstinline|x = null| \\
				\textbf{fault} but not \textbf{error} input: \lstinline|x[0] != y| \\
				\textbf{error} but not \textbf{failure} input: \lstinline|y not in x|
			\end{ex}
		\subsection{RIP model}
			\textbf{RIP model}: three things necessary to observe a failure
			\begin{enumerate}
				\item \textbf{Reachability}: PC must reach that point in the program
				\item \textbf{Infection}: after fault, program state must be incorrect
				\item \textbf{Propogation}: infected state propogates to cause bad output
			\end{enumerate}
		\subsection{Dealing with faults}
			We have three ways to deal with faults:
			\begin{itemize}
				\item \textbf{avoidance}: design, use better language
				\item \textbf{detection}: testing
				\item \textbf{tolerance}: redundancy, isolation
			\end{itemize}								
	\section{Testing}
		\subsection{Testables}
			\begin{itemize}
				\item code coverage
				\item output of a function
				\item logic coverage
				\item input space coverage
			\end{itemize}
		\subsection{Types of testing}
			\textbf{static} testing: testing without running the code
			\begin{itemize}
				\item compilation
				\item semantic verification
				\item code reviews
			\end{itemize}
			\textbf{dynamic} testing: testing by running and observing the code
			\begin{itemize}
				\item \textbf{test cases}: single input, single output (wrt to some code)
				\item \textbf{black-box testing}: don't look at system implementation
				\item \textbf{white-box testing}: base tests on system's design
			\end{itemize}
		\subsection{Coverage}
			We find a reduced space and cover that space with our tests
			\begin{description}
				\item[test requirement]: a specific element (of software) that a test case must satisfy or cover
				\item[infeasible test req]: impossible coverage e.g. unreachable code
				\item[subsumption]: when one testing criterion is strictly more powerful than another criterion
			\end{description}
			
	\section{Graph Coverage}
		\begin{description}
			\item[test path]: considering our test as some path through our program from some initial node in $N_0$, along different nodes that ends up at a final node in $N_f$
			\item[subpath]: a path which is a subsequence of a path
		\end{description}
		\subsection{Behaviours}
			\begin{itemize}	
				\item \textbf{deterministic}: 	1 test path per test case
				\item \textbf{non-deterministic}: 	multiple test paths are possible
			\end{itemize}
		\subsection{Reachability}
			\begin{itemize}
				\item \textbf{syntactically}:	reachable via edges and nodes
				\item \textbf{semantically}:	there exist input that gets to a certain node
			\end{itemize}
		\subsection{Coverage Criterion}
			\begin{description}		
				\item[Node Coverage]: for every statement (node), there must be a test case that executes it
				\item[Edge Coverage]: for every branch (edge), there must be a test case that goes through it
				\item[Edge-Pair Coverage]: for every path (length <= 2), there must be a test case
			\end{description}
		\subsection{Control Flow Graph}
			The fundamental graph for source code is the \textbf{Control Flow Graph} (CFG)
			\begin{itemize}
				\item \textbf{CFG node}: zero or more statements
				\item \textbf{CFG edge}: indicates that statements follow one another
			\end{itemize}
			Group together statements that are always consecutive into a \textbf{Basic Block}, with one entry and one exit
			
	\section{Path Coverage}
		\subsection{Definitions}
			\begin{description}
				\item[simple path]: no node appears more than once in the path (first and last can be the same)
				\item[prime path]: a simple path that is not a proper subpath of any other simple path
				\item[bridge]: an edge which, when removed, results in a disconnected graph
			\end{description}
			
		\subsection{Coverage Criterion}
			\begin{description}
				\item[Complete Path Coverage]: cover paths of all lengths
				\item[Prime Path Coverage]: cover every prime path
				\item[Single Round Trip Coverage]: at least one round trip (starts = end) path for each reachable node
				\item[Complete Round Trip Coverage]: all round trip paths for each reachable node
				\item[Specified Path Coverage]: specified set of paths
				\item[Bridge Coverage]: cover all bridges
			\end{description}
	\section{Concurrency}
		\subsection{Races}
			\begin{description}
				\item[Race] two concurrent accesses to the same memory and one of them is a write
			\end{description}
			
			Race freedom doesn't guarantee bug freedom, need to test code extra:
			\begin{itemize}
				\item run multiple times
				\item add noise
				\item Helgrind \ldots
				\item force scheduling
				\item static approaches
			\end{itemize}
		\subsection{Recursive Locks}
			If in one thread, there are two requests for one lock, the thread wait forever \\
			\lstinline|ReentrantLocks| know how many times they have been locked and need to be unlocked the same amount to liberate
			\begin{itemize}
				\item explicit \lstinline|lock()| and \lstinline|unlock()|
				\item \lstinline|trylock()|
			\end{itemize}
		\subsection{Bad Lock Usage}
			Lock and unlock must be paired, and comments must sufficiently describe conditions \\
			\textbf{Deadlocks} can occur if an interrupt uses the same lock as your program. 
			\begin{itemize}
				\item \lstinline|spin_lock_irqsave| disables interrupts locally and provides \lstinline|spin_lock| on symmetrical mulitiprocessors (\textbf{SMPs})
				\item \lstinline|spin_lock_irqrestore| restores interrupts to state when lock is acquired
			\end{itemize}
	\section{Assertions}
		\textbf{Assertion}: statement about the program that is true
		\begin{itemize}
			\item \textbf{precondition}: reasoning about the callee
			\item \textbf{postcondition}: reasoning about the caller
		\end{itemize}
		\subsection{Tools}
			\begin{description}
				\item Coverity
				\item iComment
				\item $\dots$
			\end{description}
		\subsection{Beliefs}
			Sometimes we do not know the truth (expected behaviour) but we can infer beliefs about how the code works. \\ 
			\newline
			\textbf{Must belief}: things that must be true in the code, any condtradiction is error
			\begin{lstlisting}[language=C]
x = *p / x	// p is not NULL
			// z != 0
			\end{lstlisting}
			\textbf{May belief}: things that correlate with each other in code
			\begin{lstlisting}[language=C]
A(); ... B();
A(); ... B();	// A and B may be paired
			\end{lstlisting}
			Check these beliefs as ``must beliefs'' and cross-check against must beliefs
			\begin{enumerate}
				\item record every sucessful MAY-belief as ``check''
				\item record every unsucessful as ``error''
				\item rank errors based on ``check'' : ``error'' ratio
			\end{enumerate}
		\subsection{Linters}
			Just because code compiles, doesn't mean it all variables are defined
			\begin{lstlisting}[language=Javascript]
function main(x) {
	if (x) {
		console.log('Yay');
	}
	else {
		console.log(num);
	}
}
main(true);
			\end{lstlisting}
			We can use \textbf{JSHint} to check that all top-level symbols resolve. On top of that we can use \textbf{pre-commit hooks} to make sure that all we checked into our repo passes the hooks. Improve further by forcing only master branch to go through all our pre-commit tests
	\section{Data Flow Criteria}
		Testing should take into account the data within nodes
		\textbf{$du$-pairs}: defintion-use pairs of nodes for variables
		\begin{lstlisting}[language=C]
int x = 5;	// definition
...
printf(x);	// use
		\end{lstlisting}
		If we consider the def line to be $n_0$ and the use $n_1$, then def($n_0$) = use($n_1$) = ${x}$
		\subsection{Defintions}
			\begin{description}
				\item[def-clear]: if our variable $v$ is not defined anywhere on our path 
				\item[use reached]: if our variable has a def-clear path from its defintion to use
				\item[$du$-path]: a simple path that is def-clear wrt $v$ from $n_i \leftarrow n_j$ where $v$ in def($n_i$) and use($n_j$)
					\begin{description}
						\item[def-path set]: fix a def and a variable, du($n_0$, $x$)
						\item[def-pair set]: fix a def, a use, and a variable, du($n_0$, $n_1$, $x$)
					\end{description}
			\end{description}
		\subsection{Coverage}
			\begin{description}
				\item[All-Defs Coverage]: a test case for each def to one use
				\item[All-Uses Coverage]: a test case for each def to every use
			\end{description}
		\subsection{Source Code}
			\begin{description}
				\item[Def]: x is assigned, defined as paramter in method, input to program
				\item[Use]: x occurs in an expression that the program evaluates
				\item[Reachability]: if \textit{it is possible} that the address at the def refers to the same one as the use
			\end{description}
		\subsection{Design Elements}
			Testing beyond single methods to ``design elements'' aka \textbf{integration testing}. We use \textbf{call graphs} where design elements are nodes and calls are edges
			\begin{description}
				\item[Caller]: unit that invokes callee
				\item[Actual Parameter]: value passed to callee
				\item[Formal Parameter]: placeholder for incoming value
				\item[Last-def]: the definiton that goes through to a call
				\item[First-use]: the first use in a method, it picks up the last-def definition 
				\item[Use-clear]: a path that is clear of uses, except for at the start and end
			\end{description}
	\section{Syntax-Based Testing}
		\subsection{Input Testing}
			Use grammars to validate inputs in our program and generate test inputs.
			\subsubsection{Defintions}
				\begin{description}
					\item[Start Symbol]: the first symbol we use
					\item[Non-terminal]: symbols that are defined by production rules
					\item[Terminal]: symbols not defined by production rules
					\item[Production rule]: how a symbol is defined by other symbols
				\end{description}
				\begin{ex}
					Grammar symbols:
					\begin{lstlisting}[language=C]
actions = action*	// actions: start symbol
action = dep|deb	// action: non-terminal
dep = 'deposit' account amount		
deb = 'debit' account amount		
account = digit{3}	// = digit{3}: production rule
amount = '\$'digit+'.'digit{2}
digit = [0-9]		// 0-9: non-terminals
					\end{lstlisting}
				\end{ex}
			\subsubsection{Coverage}
				\begin{description}
					\item[Terminal Symbol Coverage]: TR contains each terminal of grammar G
					\item[Production Coverage]: TR contains each production of grammar G
					\item[Derivation Coverage]: TR contains every possible string derivable from grammar G
				\end{description}
				\subsubsection{Grammar Mutation}
					Use different operators to generate invalid input:
					\begin{itemize}
						\item Non-terminal Replacement
						\item Terminal Replacement
						\item Terminal and Non-terminal Deletion
						\item Terminal and Non-terminal Duplication
					\end{itemize}
				\subsubsection{Fuzzing}
					\textbf{Fuzzing}: feed random valid characters in an effort to find bugs 

					\begin{itemize}
						\item Generation-based: generate random inputs that match your grammar
							Use different levels of inputs, from random ASCII $\rightarrow$ correct C code $\rightarrow$ model-correct C
						\item Mutation-based: randomly modify existing inputs, making sure to keep checksums updated
					\end{itemize}
			\subsection{Mutation Testing}
				Generate mutant by modifying programs and try to kill it with test cases 
				\subsubsection{Defintions}
					\begin{description}
						\item[Ground String]: a valid string in the language of the grammar
						\item[Mutation Operator]: a rule specifying synatactic variations of strings from the grammar
						\item[Mutant]: result of one application of a mutation operator to a ground string
					\end{description}
					Uninteresting Mutants:
					\begin{itemize}
						\item \textbf{Stillborn}: does not compile or immediately crashes
						\item \textbf{Trivial}: killed by almost any test case
						\item \textbf{Equivalent}: indistinguishable from original program
					\end{itemize}
				\subsubsection{Coverage}
					A test case $t$ \textbf{kills} a mutant $m$ if $t$ produces a different output on $m$ than $m_0$
					\begin{description}
						\item[Mutation Coverage]: for each mutant $m$, TR required to kill $m$
						\item[Mutation Operator Coverage]: for each mutation operator $op$, TR required to kill all mutant derived using $op$
						\item[Mutation Production Coverage]: for each mutation operator $op$ and production $p$ that it can be applied to, TR required to kill mutant from $p$
					\end{description}
				\subsubsection{Weak and Strong}
					\begin{description}
						\item[Strong Mutation]: fault must be reachable, infect state, and propogate to output
						\item[Weak Mutation]: fault which kills the mutant needs to only be reachable and infect state
						\item[Strong Mutation Coverage]: for each mutant, TR has a test that strongly kills it
						\item[Weak Mutation Coverage]: for each mutant, TR has a test that weakly kills it
					\end{description}
				\subsubsection{Mutation Operators}
					\begin{itemize}
						\item Absolute value insertion: $x > a \Rightarrow x > abs(a)$
						\item Operator replacement: $x > a \Rightarrow x < a$
						\item Scalar variable replacement: $x > a \Rightarrow x > b$
						\item Crash statement replacement: \dots
					\end{itemize}
				\subsubsection{Integration Mutation}
					\begin{itemize}
						\item Change calling method's parameters
						\item Change method being called
						\item Change inputs and outputs of called method
					\end{itemize}

							

		\section{Logic Coverage}
		\subsection{Introduction}
		Cover the \textbf{predicates} by covering the \textbf{clauses} they're made of. 
		Clauses are connected by \textbf{logical operators}:
		\begin{multicols}{3}
			\begin{itemize}
				\item $\neg$ negation
				\item $\and$ and
				\item $\lor$ or
				\item $\implies$ implication
				\item $\oplus$ exclusive or
				\item $\iff$ equivalence
			\end{itemize}
		\end{multicols}
		\subsection{Basic Coverages}
		\begin{description}
			\item[Predicate Coverage]: for each predicate $p$, cover $p = true$ and $p = false$ (analagous to CFG edge coverage)
			\item[Clause Coverage]: for each clause $c$, cover $c = true$ and $c = false$ 
			\item[Combination Coverage]: for each predicate $p$, cover all possible truth values for clauses in $C_p$ 
		\end{description}
		\subsection{Active Clause Coverage}
		We want to make a \textbf{major} clause determine the predicate, by setting the other (\textbf{minor}) clauses.
		If $p_{c=true}$ represents $p$ with $c = true$ and similarly $p_{c=false}$ then $p_c$ describes the conditions necessary
		for $c$ to determine $p$
		\begin{equation*}
			p_c = p_{c=true} \oplus p_{c=false}
		\end{equation*}

		\begin{ex}
			Determining a predicate
			\begin{align*}
				p &= a \lor b & \\
				p_a &= (true \lor b) \oplus (false \lor b) & \\
				&= true \oplus b & \\
				&= \neg b &
			\end{align*}
			Therefore $a$ determines $p$ when $b = false$
		\end{ex}

		\begin{description}
			\item[Active Clause Coverage]: for every major clause $c_i = true$ and $c_i = false$, cover all assignments of minor clauses so that $c_i$ determines $p$ 
			\item[General ACC]: ACC without restrictions on minor clauses, they may be different for $c_i = true$ and $c_i = false$
			\item[Correlated ACC]: ACC but minor clauses must cause $p = true$ for one value of $c_i$ and $p = false$ for the other
			\item[Restricted ACC]: ACC but minor clauses must be the same for $c_i = true$ and $c_i = false$
		\end{description}
		\subsection{Feasibility}
		Sometimes TRs can be infeasible, this can happen for many reasons:
		\begin{itemize}
			\item predicate is unreachable
			\item predicate never has appropriate values to cause desired clause values \textit{e.g. len(array)<0} 
			\item clause never determines predicate and you're attempting ACC
		\end{itemize}
		Therefore to get logic coverage:
		\begin{enumerate}
			\item identify the predicates 
			\item figure out how to reach each predicate
			\item make $c$ determine predicate $p$
			\item find values for program vars to meet criteria
		\end{enumerate}

		\section{Reporting Bugs}
		Bug reporting help fix bugs faster and improve software dev cycle. Good bugs that should be reported are:
		\begin{itemize}
			\item sufficiently general
			\item have severe consequences
			\item affect most recent version
		\end{itemize}
		Good reports allow developers to quickly understand and solve the bug:
		\begin{itemize}
			\item reported in the database
			\item simple: one bug per report
			\item understandable, minimal, generalizable
			\item reproducible
			\item non-judgemental
			\item not duplicates
		\end{itemize}

		\section{Input Space Partitioning}
		\subsection{Introduction}
		We don't have time to test all inputs, so we want to test one input from each representative partition. 
		\begin{description}
			\item[Characteristics]: how values are distinguished
			\item[Partition]: a way of splitting values into blocks
			\item[Block]: a set of values similar by characteristics
		\end{description}
		
		Partition should be:
		\begin{itemize}
			\item \textbf{Complete}: covers the entire domain
			\item \textbf{Disjoint}: no overlap
		\end{itemize}
		
		\subsection{Input Domain Modelling}
		\begin{enumerate}
			\item find units/functions to test
			\item identify parameters of each unit
			\item come up with input space model
		\end{enumerate}
		There are two general approaches to Input Domain Modelling:

		\textbf{Interface-based}: just using input space (without considering how it is used)
		\begin{itemize}
			\item[+] easy to identify characters
			\item[+] easy to translate test cases
			\item[-] less effective because it doesn't use domain knowledge \textit{e.g. domain knowledge}
		\end{itemize}

		\textbf{Functionality-based}: using a functional view of the program (how the program works)
		\begin{itemize}
			\item[+] may have better test cases because of domain knowledge
			\item[+] can create models just from specifications (without seeing actual code)
			\item[-] may be hard to identify characterstics, values
			\item[-] harder to generate test from this IDM
		\end{itemize}
		
		\subsection{Identifying Characteristics}
		Some possible characteristics:
		\begin{itemize}
			\item preconditions and postconditions
			\item relationships between variables 
			\item extract characteristics from specifications
		\end{itemize}

		\subsection{Choosing Blocks and Values}
		\begin{itemize}
			\item both valid and invalid values
			\item boundary and non-boundary values
			\item special values
		\end{itemize}
		An \textbf{alternate approach}, instead of multiple blocks, is to choose True/False characteristics.
		This dichotomy guarantees disjointness and completeness. \\
		We could also use \textbf{multiple IDMs}, one for valid values and one for invalid values.

		\subsection{Combining Characteristics}
		We want to test multiple inputs/characteristics, each with their own partitions 
		
		\begin{description}
			\item[All Combinations Coverage]: all combinations of blocks from all characteristics (exhuastive, but too big!)
			\item[Each Clause Coverage]: at least one value from each block for each characteristic
			\item[Pair-Wise Coverage]: combine a value for each block for each characteristic with some value from every other block
			\item[T-Wise Coverage]: same as pair-wise for T values from other blocks for other characteristics
		\end{description}

		Drop infeasible TRs

		\subsection{Base Choice}
		We have been treating all blocks as equal, but we can designate one block as the most important called the \textbf{base choice}
		and by choosing the base choice for each characteristic we have a \textbf{base test}

	\begin{description}
		\item[Base Choice Coverage]: vary the base test one characteristic at a time with all other blocks
		\item[Multiple Base Choice Coverage]: use more than one base choice block for each characteristic and then vary
	\end{description}

	If TRs are infeasible, change base case

		


	

	\section{Regression Testing}
	\subsection{Overview}
	\textbf{Regression Testing}: re-testing modified programs to uncover regression, usually with integration test
	\begin{itemize}
		\item automated: with git hooks, push process \dots
		\item appropriately sized: too small and it is useless, too large and it takes too long
		\item up-to-date: tests are valid for the version of the program being tested
	\end{itemize}

	Regression tests have a low yield for bug-finding, so automation is very important.
	\begin{itemize}
		\item input: file input is easiest, but we may need to create special mocks
		\item output: verifying output should account for resolution, whitespace \dots
		\item UI: capture and replay events, use Selenium for web dev
	\end{itemize}

	Good TODOs:
	\begin{itemize}
		\item remove redunant/old test cases over time
		\item evaluate test cases manually or with coverage criteria
		\item test case prioritization is good but implementation is unclear
	\end{itemize}

	\subsection{Best Practises}
	\begin{description}
		\item[Unit Tests]: each class has associated unit tests
		\item[Code Reviews]: code must be approved by ``owners'' of that branch/block
		\item[Continuous Builds]: continuously checkout and build latest code to enforce good practises
		\item[One-button Deploy]: if all test have passed, deploying should be easy
		\item[Back Button]: systems should have easy rollback in case of errors
	\end{description}

	\section{Badly Designed Tests}
	\textbf{Smell}: a symptom of a problem implying something is wrong

	\subsection{Project Smells}
	Highest level smells, usually detected by project managers
	\begin{itemize}
		\item production bugs: too many production bugs implies bad testing, development process, resources \dots
		\item continuous integration failures: could be buggy, expensive, or insufficient tests
	\end{itemize}

	\subsection{Behaviour Smells}
	Typically manifest as compile errors or test failures
	\begin{itemize}
		\item fragile tests: tests too sensitive to interface, data, context, or behaviour
		\item assertion roulette: integration failures with bad error messages
		\item erratic/flakey tests: could be caused by external dependencies
		\item frequent debugging: insufficient or not fine enough tests
		\item slow tests
		\item tests requiring manual intervention
	\end{itemize}

	\subsection{Code Smells}
	Affect maintenance costs and are also early warning signs of behaviour smells

	\begin{itemize}
		\item obscure tests: caused by poorly-named or implemented tests
		\item conditional test logic
		\item hard-coded test data
		\item hard-to-test code
		\item test code duplication
		\item test logic in production
	\end{itemize}


\end{document}

