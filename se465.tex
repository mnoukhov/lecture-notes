\documentclass[]{article}
\usepackage{etex}
\usepackage[margin = 1.5in]{geometry}
\setlength{\parindent}{0in}
\usepackage{amsmath}
\usepackage{amsfonts}
\usepackage{amssymb}
\usepackage{amsthm}
\usepackage{listings}
\usepackage{color}
\usepackage{mathtools}
\usepackage{pgfplots}
\usepackage[lined]{algorithm2e}
\usepackage{qtree}
\usepackage{xytree}
\usepackage{float}
\usepackage[T1]{fontenc}
\usepackage{ae,aecompl}
\usepackage[pdftex,
  pdfauthor={Michael Noukhovitch},
  pdftitle={SE 465: Testing},
  pdfsubject={Lecture notes from SE 465 at the University of Waterloo},
  pdfproducer={LaTeX},
  pdfcreator={pdflatex}]{hyperref}

\usepackage{cleveref}

\definecolor{dkgreen}{rgb}{0,0.6,0}
\definecolor{gray}{rgb}{0.5,0.5,0.5}
\definecolor{mauve}{rgb}{0.58,0,0.82}

\lstset{
  language=Java,
  aboveskip=3mm,
  belowskip=3mm,
  showstringspaces=false,
  columns=flexible,
  basicstyle={\small\ttfamily},
  numbers=none,
  numberstyle=\tiny\color{gray},
  keywordstyle=\color{blue},
  commentstyle=\color{dkgreen},
  stringstyle=\color{mauve},
  breaklines=true,
  breakatwhitespace=true,
  tabsize=4
}

\theoremstyle{definition}
\newtheorem*{defn}{Definition}
\newtheorem{ex}{Example}[section]
\newtheorem*{theorem}{Theorem}

\setlength{\marginparwidth}{1.5in}
\setlength{\algomargin}{0.75em}

\DeclarePairedDelimiter{\set}{\lbrace}{\rbrace}

\definecolor{darkish-blue}{RGB}{25,103,185}

\usepackage{hyperref}
\hypersetup{
    colorlinks,
    citecolor=darkish-blue,
    filecolor=darkish-blue,
    linkcolor=darkish-blue,
    urlcolor=darkish-blue
}
\newcommand{\lecture}[1]{\marginpar{{\footnotesize $\leftarrow$ \underline{#1}}}}

\makeatletter
\def\blfootnote{\gdef\@thefnmark{}\@footnotetext}
\makeatother

\begin{document}
	\let\ref\Cref

	\title{\bf{SE 465: Testing}}
	\date{Winter 2015, University of Waterloo \\ \center Notes written from Patrick Lam's lectures.}
	\author{Michael Noukhovitch}

	\maketitle
	\newpage
	\tableofcontents
	\newpage

	\section{Introduction}
		\subsection{Types of Problems}
			\begin{itemize}
				\item \textbf{fault}: static defect in the software
				\begin{itemize}
					\item \textbf{design fault}
					\item \textbf{mechanical fault}
				\end{itemize}				 
				\item \textbf{error}: have incorrect state
				\item \textbf{failure}: external incorrect behavior
			\end{itemize}
			\begin{ex}
				Faults
				\begin{lstlisting}{language=Java}
	static public int findLast (int[] x, int y) {
		for (int i=x.length-1; i>0; --i){
	    	if (x[i] == y){
	        	return i;
	        }
	   	}
	   	return -1;
	}			
				\end{lstlisting}
				
				\textbf{fault}: should be \lstinline|i >= 0| \\
				no \textbf{fault} input: \lstinline|x = null| \\
				\textbf{fault} but not \textbf{error} input: \lstinline|x[0] != y| \\
				\textbf{error} but not \textbf{failure} input: \lstinline|y not in x|
			\end{ex}
		\subsection{RIP model}
			\textbf{RIP model}: three things necessary to observe a failure
			\begin{enumerate}
				\item \textbf{Reachability}: PC must reach that point in the program
				\item \textbf{Infection}: after fault, program state must be incorrect
				\item \textbf{Propogation}: infected state propogates to cause bad output
			\end{enumerate}
		\subsection{Dealing with faults}
			We have three ways to deal with faults:
			\begin{itemize}
				\item \textbf{avoidance}: design, use better language
				\item \textbf{detection}: testing
				\item \textbf{tolerance}: redundancy, isolation
			\end{itemize}								
	\section{Testing}
		\subsection{Testables}
			\begin{itemize}
				\item code coverage
				\item output of a function
				\item logic coverage
				\item input space coverage
			\end{itemize}
		\subsection{Types of testing}
			\textbf{static} testing: testing without running the code
			\begin{itemize}
				\item compilation
				\item semantic verification
				\item code reviews
			\end{itemize}
			\textbf{dynamic} testing: testing by running and observing the code
			\begin{itemize}
				\item \textbf{test cases}: single input, single output (wrt to some code)
				\item \textbf{black-box testing}: don't look at system implementation
				\item \textbf{white-box testing}: base tests on system's design
			\end{itemize}
		\subsection{Coverage}
			We find a reduced space and cover that space with our tests\\
			\textbf{test requirement}: a specific element (of software) that a test case must satisfy or cover
			\textbf{infeasable test requirement}: impossible coverage e.g. unreachable code \\
			\textbf{subsumption}: when one testing criterion is strictly more powerful than another criterion
	\section{Graph Coverage}
		\textbf{test path}: considering our test as some path through our program from some initial node in $N_0$, along different nodes that ends up at a final node in $N_f$ \\
		\textbf{subpath}: a path which is a subsequence of a path
		\subsection{Behaviours}
			\begin{itemize}	
				\item \textbf{deterministic}: 	1 test path per test case
				\item \textbf{non-deterministic}: 	multiple test paths are possible
			\end{itemize}
		\subsection{Reachability}
			\begin{itemize}
				\item \textbf{syntactically}:	reachable via edges and nodes
				\item \textbf{semantically}:	there exist input that gets to a certain node
			\end{itemize}
		\subsection{Coverage Criterion}
			\textbf{Node Coverage}: for every statement (node), there must be a test case that executes it
			\textbf{Edge Coverage}: for every branch (edge), there must be a test case that goes through it
			\textbf{Edge-Pair Coverage}: for every path of length up to 2, there must be a test case that goes through it
	\section{Paths}
		\subsection{Definitions}
			\textbf{simple path}: no node appears more than once in the path (but first and last can be the same) \\
			\textbf{prime path}: a simple path that is not a proper subpath of any other simple path
		\subsection{Coverage Criterion}
			\textbf{Complete Path Coverage}: test cases cover paths of all lengths
			\textbf{Prime Path Coverage}: a test case for every prime path
			
\end{document}
