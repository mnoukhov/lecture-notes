\documentclass[]{article}
\usepackage{etex}
\usepackage[margin = 1.5in]{geometry}
\setlength{\parindent}{0in}
\usepackage{amsmath}
\usepackage{amsfonts}
\usepackage{amssymb}
\usepackage{amsthm}
\usepackage{listings}
\usepackage{color}
\usepackage{mathtools}
\usepackage{multicol}
\usepackage{pgfplots}
\usepackage{qtree}
\usepackage{xytree}
\usepackage[lined]{algorithm2e}
\usepackage{float}
\usepackage[T1]{fontenc}
\usepackage{ae,aecompl}
\usepackage[pdftex,
  pdfauthor={Michael Noukhovitch},
  pdftitle={EDA040: Concurrent Programming},
  pdfsubject={Lecture notes from EDA040 at the Lund University},
  pdfproducer={LaTeX},
  pdfcreator={pdflatex}]{hyperref}

\usepackage{cleveref}
\usepackage{enumitem}

\definecolor{dkgreen}{rgb}{0,0.6,0}
\definecolor{gray}{rgb}{0.5,0.5,0.5}
\definecolor{mauve}{rgb}{0.58,0,0.82}

\lstset{
  language=Java,
  aboveskip=3mm,
  belowskip=3mm,
  showstringspaces=false,
  columns=flexible,
  basicstyle={\small\ttfamily},
  numbers=none,
  numberstyle=\tiny\color{gray},
  keywordstyle=\color{blue},
  commentstyle=\color{dkgreen},
  stringstyle=\color{mauve},
  breaklines=true,
  breakatwhitespace=true,
  tabsize=4
}

\theoremstyle{definition}
\newtheorem*{defn}{Definition}
\newtheorem{ex}{Example}[section]
\newtheorem*{theorem}{Theorem}

\setlength{\marginparwidth}{1.5in}
\setlength{\algomargin}{0.75em}

\DeclarePairedDelimiter{\set}{\lbrace}{\rbrace}

\definecolor{darkish-blue}{RGB}{25,103,185}

\usepackage{hyperref}
\hypersetup{
    colorlinks,
    citecolor=darkish-blue,
    filecolor=darkish-blue,
    linkcolor=darkish-blue,
    urlcolor=darkish-blue
}
\newcommand{\lecture}[1]{\marginpar{{\footnotesize $\leftarrow$ \underline{#1}}}}

\makeatletter
\def\blfootnote{\gdef\@thefnmark{}\@footnotetext}
\makeatother

\begin{document}
	\let\ref\Cref

	\title{\bf{EDA040: Concurrent Programming}}
	\date{Fall 2015, Lund University\\ \center Notes written from Klas Nilsson's lectures.}
	\author{Michael Noukhovitch}

	\maketitle
	\newpage
	\tableofcontents
	\newpage

	\section{Introduction}
	\subsection{Concurrency}
	\begin{description}
		\item[activity] entity performing actions
		\item[process] entity performing instructions with own resources
		\item[job] sequential instructions to be performed by an activity
		\item[task] a set of jobs being performed by some process
		\item[thread] sequential activity performing instructions
	\end{description}

	\subsection{Threads}
	\begin{description}
		\item[execution thread] the thread itself accessed via the \lstinline|Thread| interface
	\end{description}

	\section{Mutual Exclusion}
	\subsection{Requirements}
	\begin{itemize}
		\item mutual exclusion
		\item no deadlock
		\item no starvation
		\item efficiency
	\end{itemize}
	\subsection{Semaphores}
	\textbf{semaphore} simple counting interface for concurrency
	\subsubsection{Mutex}
	used to lock and unlock critical sections
	\begin{lstlisting}[language=Java]
MutexSem mutex;
mutex.take()
// critical section
mutex.give()
	\end{lstlisting}
	\subsubsection{Signaling}
	calls used to block or unblock a thread
	\begin{lstlisting}[language=Java]
CountingSem mutex = new CountingSem();
// thread A
*
mutex.take() // block this thread
*
// thread B
*
mutex.give() // unblock thread A
*
	\end{lstlisting}

	\subsubsection{Other Types}
	\begin{itemize}
		\item blocked-set: 
		\item blocked-queue
		\item blocked-priority
		\item binary semaphore
		\item multi-step semaphore
	\end{itemize}<++>




	

\end{document}
