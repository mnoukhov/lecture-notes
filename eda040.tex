\documentclass[]{article}
\usepackage{etex}
\usepackage[margin = 1.5in]{geometry}
\setlength{\parindent}{0in}
\usepackage{amsmath}
\usepackage{amsfonts}
\usepackage{amssymb}
\usepackage{amsthm}
\usepackage{listings}
\usepackage{color}
\usepackage{mathtools}
\usepackage{multicol}
\usepackage{pgfplots}
\usepackage{qtree}
\usepackage{xytree}
\usepackage[lined]{algorithm2e}
\usepackage{float}
\usepackage[T1]{fontenc}
\usepackage{ae,aecompl}
\usepackage[pdftex,
  pdfauthor={Michael Noukhovitch},
  pdftitle={EDA040: Concurrent Programming},
  pdfsubject={Lecture notes from EDA040 at the Lund University},
  pdfproducer={LaTeX},
  pdfcreator={pdflatex}]{hyperref}

\usepackage{cleveref}
\usepackage{enumitem}

\definecolor{dkgreen}{rgb}{0,0.6,0}
\definecolor{gray}{rgb}{0.5,0.5,0.5}
\definecolor{mauve}{rgb}{0.58,0,0.82}

\lstset{
  language=Java,
  aboveskip=3mm,
  belowskip=3mm,
  showstringspaces=false,
  columns=flexible,
  basicstyle={\small\ttfamily},
  numbers=none,
  numberstyle=\tiny\color{gray},
  keywordstyle=\color{blue},
  commentstyle=\color{dkgreen},
  stringstyle=\color{mauve},
  breaklines=true,
  breakatwhitespace=true,
  tabsize=4
}

\theoremstyle{definition}
\newtheorem*{defn}{Definition}
\newtheorem{ex}{Example}[section]
\newtheorem*{theorem}{Theorem}

\setlength{\marginparwidth}{1.5in}
\setlength{\algomargin}{0.75em}

\DeclarePairedDelimiter{\set}{\lbrace}{\rbrace}

\definecolor{darkish-blue}{RGB}{25,103,185}

\usepackage{hyperref}
\hypersetup{
    colorlinks,
    citecolor=darkish-blue,
    filecolor=darkish-blue,
    linkcolor=darkish-blue,
    urlcolor=darkish-blue
}
\newcommand{\lecture}[1]{\marginpar{{\footnotesize $\leftarrow$ \underline{#1}}}}

\makeatletter
\def\blfootnote{\gdef\@thefnmark{}\@footnotetext}
\makeatother

\begin{document}
	\let\ref\Cref

	\title{\bf{EDA040: Concurrent Programming}}
	\date{Fall 2015, Lund University\\ \center Notes written from Klas Nilsson's lectures.}
	\author{Michael Noukhovitch}

	\maketitle
	\newpage
	\tableofcontents
	\newpage

	\section{Introduction}
	\subsection{Concurrency}
	\begin{description}
		\item[activity] entity performing actions
		\item[process] entity performing instructions with own resources
		\item[job] sequential instructions to be performed by an activity
		\item[task] a set of jobs being performed by some process
		\item[thread] sequential activity performing instructions
		\item[execution thread] the thread itself accessed via the \lstinline|Thread| interface
	\end{description}

	\subsection{Mutual Exclusion}
	requirements:
	\begin{itemize}
		\item mutual exclusion
		\item no deadlock
		\item no starvation
		\item efficiency
	\end{itemize}
	\section{Semaphores}
	\textbf{semaphore} simple counting interface for concurrency
	\subsection{Mutex}
	used to lock and unlock critical sections
	\begin{lstlisting}[language=Java]
MutexSem mutex;
mutex.take()
// critical section
mutex.give()
	\end{lstlisting}
	\subsection{Signaling}
	calls used to block or unblock a thread
	\begin{lstlisting}[language=Java]
CountingSem mutex = new CountingSem();
// thread A
*
mutex.take() // block this thread
*
// thread B
*
mutex.give() // unblock thread A
*
	\end{lstlisting}

	\subsection{Other Types}
	\begin{itemize}
		\item blocked-set: arbitrary thread \lstinline|take|s 
		\item blocked-queue: \lstinline|take| in FIFO order
		\item blocked-priority: highest priority \lstinline|take|
		\item binary semaphore: efficient mutex implementation in RTOS
		\item multi-step semaphore: reserve several resources at once
	\end{itemize}

	\section{Monitors}
	\subsection{Introduction}
	As opposed to using \lstinline|take/give| throughout the program, we instead can limit our mutual exlcusions to specific function. \\
	\textbf{Monitor}: interface for mutually exclusive access to a function, in java using \lstinline|synchronized|
	\begin{itemize}
		\item \lstinline|wait| stateless wait for signal
		\item \lstinline|notify| notify first (or highest prio) waiting task
		\item \lstinline|notifyAll| notify all waiting task
	\end{itemize}
	\subsection{Rules}
	\begin{itemize}
		\item don't mix a thread and monitor
		\item all public methods should be synchronized
		\item wrap thread-unsafe classes by monitor
		\item don't use (spread-out) synchronized blocks
	\end{itemize}

	\section{Deadlock}
	\subsection{Introduction}
	\begin{description}
		\item[deadlock] a circular chain of tasks trying to allocate resources
		\item[starvation] when a task is never prioritized to execute
		\item[livelock] a running circular chain that is unable to allocate resources
	\end{description}

	Deadlock \textbf{detection} is not feasible as a resolution for real-time applications so we will looks at \textbf{prevention} which deals with eliminating one of the conditions for deadlock:
	\begin{itemize}
		\item mutual exclusion
		\item hold and wait
		\item no preemption
		\item circular wait
	\end{itemize}

	\subsection{Dining Philosophers}
	Five dining philosophers with five forks between them but each needs two to eat, proves to be a deadlock-able situation if they circularly pick up one fork. We can solve it with: 
	\begin{itemize}
		\item One left-handed philosopher that picks up left fork first (not circular)
		\item Only allowing four philosophers into the room at a time (monitor)
		\item Philosophers picking up both forks or neither (using a \lstinline|Multistep Sem|, starvation possible)
	\end{itemize}

	\section{Message-based Synchronization}
	\subsection{Mailboxes}
	Message-based communication is useful for:
	\begin{itemize}
		\item producer-consumer relations
		\item signaling (one thread never waits)
		\item information transfer (in data of message)
		\item buffering
		\item distributed concurrency
		\item encapsulation (concurrent object properties)
	\end{itemize}

	\subsection{Unbounded mailbox}
	Use copy-on-send to create limitless mailboxes
	\begin{itemize}
		\item[+] flexible code
		\item[+] no need to assume shared memory
		\item[+] thread safety, message not accessible by sender
		\item[-] can run out of memory, increased memory use
		\item[-] unpractical when immediate response is required
		\item[-] recycling via message pools is difficult
	\end{itemize}

	\subsection{Implementation}
	We will use \lstinline|java.util.EventObject| as our message and \lstinline|RTEvent| for async communication because it includes a timestamp. As such we will use \lstinline|RTEventBuffer| as our circular mailbox
\end{document}
